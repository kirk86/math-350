\documentclass[10pt,letterpaper]{article}
\renewcommand{\rmdefault}{ptm}

\usepackage[left=1in,right=1in,top=1in,bottom=1in]{geometry} 
\usepackage{amsmath}
\usepackage{amsfonts}
\usepackage{amsthm}
\usepackage{amssymb}
\usepackage{polynomial}
\usepackage{layouts}

\usepackage{enumerate}

\usepackage{syntax}
\usepackage{gensymb}
\usepackage{cancel}
\usepackage{calc}
\usepackage{enumerate}
\usepackage{xcolor}

\usepackage{minted}

\usepackage[version=0.96]{pgf}
\usepackage{tikz}
\usetikzlibrary{arrows,shapes,automata,backgrounds,petri,positioning}
\usetikzlibrary{decorations.pathmorphing}
\usetikzlibrary{decorations.shapes}
\usetikzlibrary{decorations.text}
\usetikzlibrary{decorations.fractals}
\usetikzlibrary{decorations.footprints}
\usetikzlibrary{shadows}
\usetikzlibrary{calc}
\usetikzlibrary{spy}
\usetikzlibrary{matrix}

\usepackage{tikz-qtree}

\setcounter{tocdepth}{2}
\setcounter{secnumdepth}{4}
\usepackage[bookmarksopen,bookmarksdepth=3]{hyperref}
\usepackage{titlesec}


%define new colors
\definecolor{dark-red}{rgb}{0.8,0.15,0.15}
\definecolor{dark-blue}{rgb}{0.15,0.15,0.7}
\definecolor{medium-blue}{rgb}{0,0,0.5}
\definecolor{dark-green}{rgb}{0.2,0.7,0.7}

%set up color for table of contents
\hypersetup{
    colorlinks, linkcolor={dark-green},
    citecolor={dark-blue}, urlcolor={medium-blue}
}

\usepackage{tocloft}

%preven linebreak between subsection header and its content
\titleformat{\subsection}[runin]{\normalfont\bfseries}{\thesubsection.}{2pt}{}
%\titleformat{\section}[runin]{\normalfont\bfseries\filcenter}{\thesection.}{5pt}{}


\titleformat{\section}[block]
{\normalfont\sffamily\LARGE}
{\thesection}{.2em}{\titlerule\\[.2ex]\bfseries}

%title
\title{\textbf{Math 350 - Advanced Calculus \\ Homework 6}}
\author{Chan Nguyen}

%set numwidth of section
\setlength{\cftsecnumwidth}{1.5cm} 
%make subsection numwidth different than as section
\setlength{\cftsubsecnumwidth}{3cm}
%make subsection indent the same as section
\setlength{\cftsubsecindent}{\cftsecindent} 

\newcommand{\sol}{\textbf{Solution}}

\usepackage{tikz}
\usetikzlibrary{matrix}
\usetikzlibrary{shapes,backgrounds}

\makeatletter
\newcommand{\DESCRIPTION@original@item}{}
\let\DESCRIPTION@original@item\item
\newcommand*{\DESCRIPTION@envir}{DESCRIPTION}
\newlength{\DESCRIPTION@totalleftmargin}
\newlength{\DESCRIPTION@linewidth}
\newcommand{\DESCRIPTION@makelabel}[1]{\llap{#1}}%
\newcommand{\DESCRIPTION@item}[1][]{%
  \setlength{\@totalleftmargin}%
       {\DESCRIPTION@totalleftmargin+\widthof{\textbf{#1 }}-\leftmargin}%
  \setlength{\linewidth}
       {\DESCRIPTION@linewidth-\widthof{\textbf{#1 }}+\leftmargin}%
  \par\parshape \@ne \@totalleftmargin \linewidth
  \DESCRIPTION@original@item[\textbf{#1}]%
}
\newenvironment{DESCRIPTION}
  {\list{}{\setlength{\labelwidth}{0cm}%
           \let\makelabel\DESCRIPTION@makelabel}%
   \setlength{\DESCRIPTION@totalleftmargin}{\@totalleftmargin}%
   \setlength{\DESCRIPTION@linewidth}{\linewidth}%
   \renewcommand{\item}{\ifx\@currenvir\DESCRIPTION@envir
                           \expandafter\DESCRIPTION@item
                        \else
                           \expandafter\DESCRIPTION@original@item
                        \fi}}
  {\endlist}
\makeatother

\begin{document}

\tableofcontents 
\maketitle

\setlength{\parindent}{0pt}
\setlength{\parskip}{1ex}
	\phantomsection
	\subsection*{{\color{purple}\underline{Problem 1}}}
	\addcontentsline{toc}{subsection}{\numberline{}Problem 1}
	Let $0 < a < 3$ be a number. Let $x_0 = a$ and $y_0 = a - 1$, and inductively define
	$$x_{n+1} = x_n - \dfrac{x_ny_n}{2}$$
	$$y_{n+1} = \dfrac{y_n^2(y_n - 3)}{4}$$
	\begin{enumerate}[(a)]
		\item Prove that $a(1 + y_n) = x_n^2$ for all $n$.
		\item Prove that the sequence $(y_n)$ converges to $0$.
		\item Prove that the sequence $(x_n)$ converges to $\sqrt{a}$
	\end{enumerate}
	\textbf{Solution. }
	\begin{enumerate}[(a)]
		\item
		\begin{proof} 
		We will prove by induction $n$. Base case:
		$$\text{lhs} = x_0^2 = a^2 = \text{rhs} = a(1 + y_0) = a(1 + a - 1) = a^2$$
		Suppose $a(1 + y_n) = x_n^2$, we will show that it's also true for $n + 1$.
		We have
		\begin{eqnarray*}
		x^2_{n+1} &=& \bigg(x_n - \dfrac{x_ny_n}{2}\bigg)^2 \\
			    &=& x_n^2 - 2\cdot x_n \cdot \dfrac{x_ny_n}{2} + \dfrac{x_n^2y_n^2}{4} \\
		&=& x_n^2(1 - y_n + \dfrac{y_n^2}{4}) \\
		&=& a(1 + y_n)(1 - y_n + \dfrac{y_n^2}{4}) \\
		&=& a(1 - y_n^2 + \dfrac{y_n^2}{4} + \dfrac{y_n^3}{4}) \\
		&=& a(1 - \dfrac{3y_n^2}{4} + \dfrac{y_n^3}{4}) \\
		&=& a(1 + \dfrac{y_n^2(y_n - 3)}{4}) \\
		&=& a(1 + y_{n+1})  
		\end{eqnarray*}
		
		\end{proof}
		\item
		\begin{proof} 
		Since $0 < a < 3 \Rightarrow -1 < a - 1 = y_0 < 2$. A brief C++ program shows that:
\begin{verbatim}
a = -0.5
------------------	
-0.5
-0.21875
-0.0385056
-0.00112628
-9.51739e-007
-6.79356e-013
-3.46143e-025
-8.98612e-050
-6.05628e-099
-2.75089e-197
-0
-0
-0
-0
-0
\end{verbatim}
\begin{verbatim}
initial a = 0.5
------------------
0.5
-0.15625
-0.0192642
-0.00028012
-5.88559e-008
-2.59801e-015
-5.06225e-030
-1.92198e-059
-2.77049e-118
-5.75673e-236
-0
-0
-0
-0
-0
\end{verbatim}
\begin{minted}{c++}
#include <iostream>
#include <vector>
#include <algorithm>

using namespace std;

void generate_sequence(double init) {
	cout << "initial a = " << init << endl;
	cout << "------------------\n";
	double y0 = init;
	for (int i = 0; i < 15; ++i) {
		cout << y0 << endl;
		y0 = y0*y0*(y0 - 3)/4.0;
	}
	cout << endl << endl;
}

int main() {
	double inits[5] = {-0.5, 0.0, 0.5, 1.0, 1.5};
	for (int i = 0; i < 5; ++i) {
		generate_sequence(inits[i]);
	}
	return 0;
}
\end{minted}
	As we can see if the initial value falls in the range between $(-1, 2)$ , the sequence is actually increasing 
	and converges to $0$ except for the first value. So our goal is to prove that it's bounded by $0$ and increasing.
	First we will prove that $(y_n)$ is bounded by $0$ for all $n \geq 1$. Base case $n = 1$, we have
	$$y_1 = \dfrac{y_0^3 - 3y_0^2}{4}$$
	where $-1 < y_0 < 2 \Rightarrow -1 < y_0^3 < 8$, and $0 \leq 3y_0^2 < 12$. Thus $\dfrac{y_0^3 - 3y_0^2}{4} < 0$.
	Suppose that $y_n < 0$, we want to show that it's also true for $n + 1$. Indeed,
	$y_{n+1} = \dfrac{y_n^3 - 3y_n^2}{4} < 0$ because $y_n^3 < 0$ and $3y_n^2 > 0$. Therefore,
	$(y_n)$ is bounded above by $0$. \\
	Next we want to show that $(y_n)$ is increasing starting from $y_1$, so we want to show
	that
	\begin{eqnarray*}
		& & y_{n+1} \geq y_n \, \, \, ,\forall n \\
		&\Leftrightarrow & \dfrac{y_n^2(y_n - 3)}{4} \geq y_{n} \\
		&\Leftrightarrow & y_n^3 - 3y_n^2 - 4y_n \geq 0 
	\end{eqnarray*}	 
	We have $$y(a) = y_n^3 - 3y_n^2 - 4y_n = y_n(y_n - 4)(y_n + 1)$$ where $y_n \leq 0$.  
	$$\underbrace{y_n}_{\leq 0}\underbrace{(y_n - 4)}_{\leq 0}\underbrace{(y_n + 1)}_{???}$$
 	To make this expression greater than 0, we need $y_n \geq -1$ which is true because 
 	$y_n \leq 0$ for all $n$, so it will never be able to reach $-1$ except for the initial
 	value which is consistent with our data from the program. Hence $(y_n)$ is bounded by $0$
 	and it's increasing after the first few terms, so there exists $N \in \mathbb{N}$ such that
 	if $n > N$ then $(y_n)$ is bounded and increasing. By Monotone Theorem, we can conclude
 	that $(y_n)$ converges or the limit of $(y_n)$ exists. 
	To show that this limit is $0$, let $L$ be the limit of the sequence. We have
	$$\displaystyle\lim_{n\to\infty}y_n = \displaystyle\lim_{n\to\infty} = L$$
	Hence,
	\begin{eqnarray*}
		&& \dfrac{L^2(L - 3)}{4} = L \\
		&\Leftrightarrow & L^3 - 3L - 4L = 0 \\
		&\Leftrightarrow & L(L - 4)(L + 1) = 0 \\
	\end{eqnarray*}
	There are 3 solutions to this equation, however only $0$ actually works. $4$ can be eliminated because 
	$4 \not\in (-1, 2)$ and we know that the $(y_n)$ is bounded by $0$. The same logic applies to $-1$,
	since the sequence is bounded by $0$ and increasing, it can't be $-1$. Therefore the limit of $(y_n)$
	must be $0$.
	\end{proof}
	
	\item
	\begin{proof} 
	From part (a) we have that:
	$$x_n^2 = a(1 + y_n) \Leftrightarrow x_n = \sqrt{a(1 + y_n)}$$ 
	Hence,
	$$\displaystyle\lim_{n\to\infty}x_n = \displaystyle\lim_{n\to\infty} \sqrt{a(1 + y_n)}$$
	By Algebra Limit Theorem, we have
	\begin{eqnarray*}
		\displaystyle\lim_{n\to\infty} \sqrt{a(1 + y_n)} &=&
		\displaystyle\lim_{n\to\infty} \sqrt{a}  + \displaystyle\lim_{n\to\infty}ay_n \\
		&=& \sqrt{a} + a \cdot 0 \\
		&=& \sqrt{a}
	\end{eqnarray*}
	\end{proof}

	\end{enumerate}
	
	\phantomsection
	\subsection*{{\color{purple}\underline{Problem 2}}}
	\addcontentsline{toc}{subsection}{\numberline{}Problem 2}
	Let $S \subset \mathbb{R}$. A number $x$ is an interior point of the set $S \subset R$ if there is $r > 0$
	such that the interval $(x - r, x + r) \subset S$. The set of all interior points of $S$ is denoted by $S^{\circ}$.
	Prove the following properties:
	\begin{enumerate}[(i)]
		\item $S^{\circ} \subset S$
		\item $(S \cap T)^{\circ} = S^{\circ} \cap T^{\circ}$	
		\item $S^{\circ} \cup T^{\circ} \subset (S \cup T)^{\circ}$, but these two sets are not necessarily equal.
		\item $S^{\circ}$ is the largest open set contained in $S$, that is $S^{\circ}$ is an open set, and if
		$U \subset S$ is an open set, then $U \subset S^{\circ}$.
	\end{enumerate}
	\textbf{Solution. }
	\begin{enumerate}[(i)]
		\item Let $x \in S^{\circ}$, then by definition of interior point,
		there exists $r > 0$ such that $(x - r, x + r) \subset S \Rightarrow
		x \in S \Rightarrow S^{\circ} \subset S$.
		
		\item $\subset:$ Let $x \in (S \cap T)^{\circ}$, then by definition of interior point, there exists
		$r > 0$ such that $(x - r, x + r) \subset (S \cap T)$ which implies
		$$(x - r, x + r) \subset S$$
		and 
		$$(x - r, x + r) \subset T$$
		Hence, $x \in S^{\circ}$ and $x \in T^{\circ} \Rightarrow x \in (S^{\circ} \cap T^{\circ})
		\Rightarrow (S \cap T)^{\circ} \subset (S^{\circ} \cap T^{\circ})$. (1)\\
		$\supset:$ Let $x \in (S^{\circ} \cap T^{\circ}) \Rightarrow
		x \in S^{\circ}$ and $x \in T^{\circ}$. By definition of interior point, $\exists r_1, r_2 > 0$ such 
		that
		$$(x - r_1, x + r_1) \subset S \text{ and } (x - r_2, x + r_2) \subset T$$
		Let $r = \min(r_1, r_2) \Rightarrow (x - r, x + r) \subset (S \cap T) \Rightarrow
		x \in (S \cap T)^{\circ}$. (2) \\
		From $(1)$ and $(2)$ we have $(S \cap T)^{\circ} = S^{\circ} \cap T^{\circ}$
		
		\item Let $x \in (S^{\circ} \cup T^{\circ}) \Rightarrow x \in S^{\circ}$ or $x \in T^{\circ}$.
		By definition of interior point, and without loss of generality, we assume that there exists
		$r > 0$ in such that $(x - r, x + r) \subset S \Rightarrow x \in (S \cup T) \Rightarrow
		x \in (S \cup T)^{\circ}$. Thus $(S^{\circ} \cup T^{\circ}) \subset (S \cup T)^{\circ}$. \\
		However, it's not always true that $(S \cup T)^{\circ} \subset (S^{\circ} \cup T^{\circ})$.
		For example if $S = (-\infty, 0)$ and $T = (0, +\infty)$
		
		\item Recall
		\newtheorem*{tm}{Theorem 7.3}
		\begin{tm}
		A set $S \in \mathbb{R}$ is open if and only if $\forall x \in S$, $\exists r > 0$ such that
		$(x - r, x + r) \subset S$.
		\end{tm}
		By definition of interior sets, $\forall x \in S^{\circ}$, $\exists r > 0$ such that
		$(x - r, x + r) \subset S^{\circ}$. Apply Theorem 7.3 to $S^{\circ}$, we have that $S^{\circ}$ is open.
		Let $U$ be a set such that $U \subset S$
		and $U$ is open, then $\forall x \in U$, $\exists r > 0$ such that $(x - r, x + r) \subset U$,
		but $U \subset S$, thus $\forall x \in U$, $\exists r > 0$ such that $(x - r, x + r) \subset U \subset S$.
		In other words, $U$ is also a set that contains interior points of $S$, where $S^{\circ}$ is the 
		set that contains "all" interior points of $S$, thus $U \subset S^{\circ}$. Since $U$ is arbitrarily chosen,
		$S^{\circ}$ is the largest open set contained in $S$. 
		 
	\end{enumerate}
	
	\phantomsection
	\subsection*{{\color{purple}\underline{Problem 3}}}
	\addcontentsline{toc}{subsection}{\numberline{}Problem 3}
	The closure of a set of numbers $S \subset \mathbb{R}$ is the set $S^{-} = S \cup S'$, the union of the set
	and its set of limit points $S'$. Prove the following properties.
	\begin{enumerate}[(i)]
		\item If $S \subset T$ then $S^{-} \subset T^{-}$
		\item If $(S \cup T)^{-} = S^{-} \cup T^{-}$
		\item If $(S \cap T)^{-} \subset S^{-} \cap T^{-}$
		\item $S^{-}$ is the smallest closed set that contains $S$.
	\end{enumerate}
	
	\phantomsection
	\subsection*{{\color{purple}\underline{Problem 4}}}
	\addcontentsline{toc}{subsection}{\numberline{}Problem 4}
	Define $\displaystyle\lim_{x\to\ a}f(x) = \infty$ to mean that for every $N$ there is
	a $\delta > 0$ such that for all $x$ if $0 < |x - a| < \delta$ then $f(x) > N$. Prove that 
	$$\displaystyle\lim_{x\to\ 3}\dfrac{1}{(x - 3)^2} = \infty$$
	\textbf{Solution.}
	\begin{proof}
		Let $N \in \mathbb{R+}$, and $\delta = \dfrac{1}{\sqrt{N}} > 0$. Thus if 
		$0 < |x - 3| < \dfrac{1}{\sqrt{N}} \Rightarrow (|x - 3|)^2 < \dfrac{1}{N}
		\Rightarrow \dfrac{1}{(x - 3)^2} > N$. Since $N$ is arbitrarily chosen,
		$\dfrac{1}{(x - 3)^2} > N$ for all $N$. Therefore,
		$$\displaystyle\lim_{x\to\ 3}\dfrac{1}{(x - 3)^2} = \infty$$
	\end{proof}
	
	\phantomsection
	\subsection*{{\color{purple}\underline{Problem 5}}}
	\addcontentsline{toc}{subsection}{\numberline{}Problem 5} 
	Prove that if $\displaystyle\lim_{x\to\ 0} \dfrac{f(x)}{x} = L$ and $b \neq 0$, then 
	$\displaystyle\lim_{x\to\ 0} \dfrac{f(bx)}{x} = bL$ \\
	\textbf{Solution. }
	\begin{proof}
		By definition of limit, we have that there exists $\epsilon > 0$, and $\delta > 0$
		such that if $|x - 0| = |x| < \delta$ then 
		$$\bigg|\dfrac{f(x)}{x} - L\bigg| < \epsilon$$
		Now consider,
		$$\displaystyle\lim_{x\to\ 0} \dfrac{f(bx)}{x} 
		= \displaystyle\lim_{x\to\ 0} \dfrac{bf(bx)}{bx}
		= b \displaystyle\lim_{x\to\ 0} \dfrac{f(bx)}{bx}
		\text{ since } b \neq 0$$
		On the other hand, $|x| < \delta \Rightarrow |x| < \dfrac{\delta}{|b|}$ because $\dfrac{\delta}{|b|} > 0
		\Rightarrow |bx| < \delta \Rightarrow \bigg|\dfrac{f(bx)}{bx} - L\bigg| < \epsilon$. \\
		Therefore,
		$$b \displaystyle\lim_{x\to\ 0} \dfrac{f(bx)}{bx} = bL$$
	\end{proof}
	
	\phantomsection
	\subsection*{{\color{purple}\underline{Problem 6}}}
	\addcontentsline{toc}{subsection}{\numberline{}Problem 6} 
	Let $S$ be a nonempty subset of real numbers that is bounded above but has no greatest element.
	Prove that $\sup(S)$ is a limit point of $S$.
	\begin{proof}
	
	\end{proof}
	
	\phantomsection
	\subsection*{{\color{purple}\underline{Problem 7}}}
	\addcontentsline{toc}{subsection}{\numberline{}Problem 7}
	A sequence $(a_n)$ is a Cauchy sequence if, for any $\epsilon > 0$, there is a natural number $N$
	such that if $n, m > N$, then $|a_n - a_m| < \epsilon$. Prove the following:
	\begin{enumerate}[(a)]
		\item Any convergent sequence is a Cauchy sequence.
		\item Any Cauchy sequence is bounded.
		\item Any subsequence of a Cauchy sequence is a Cauchy sequence.
		\item If a subsequence of Cauchy sequence converges, then the whole sequence also converges.	
	\end{enumerate}
	\textbf{Solution. }
	\begin{enumerate}[(a)]
		\item 
		\begin{proof}
			Let $(a_n)$ be a convergent sequence to $L$, then $\forall \epsilon > 0, \exists N \in \mathbb{N}$ 
			such that if $n > N$, then $|a_n - L| < \dfrac{\epsilon}{2}$. 
			Given $\epsilon > 0$, we choose $n, m > N$, then 
			$$|a_n - L| < \dfrac{\epsilon}{2} \text{ and } 
			|a_m - L| < \dfrac{\epsilon}{2}$$
			Consider, $|a_n - a_m| = |a_{n} - L + L - a_{m}|
			< |a_{n} - L| + |a_{m} - L|$ by Triangle Inequality. Hence,
			$$|a_{n} - a_{m}| = |a_{n} - L| + |a_{m} - L| < \dfrac{\epsilon}{2} + \dfrac{\epsilon}{2} = \epsilon$$ 
			And this shows that a convergent sequence satisfy all Cauchy sequence property which implies
			it must be a Cauchy sequence.
		\end{proof}
		
		\item 
		\begin{proof}
			Let $(a_n)$ be such a Cauchy sequence, then by definition for any $\epsilon > 0$, there is a natural number $N$ such that if $n, m > N$, then $|a_n - a_m| < \epsilon$. Let $m = N + 1, n > m$ and $\epsilon = 1$, then
			$$1 > |a_n - a_{N+1}| > |a_n| - |a_{N+1}| \Leftrightarrow
			|a_n| < (|a_{N+1}| + 1) \text{ for all } n$$
			Let $M = \max\{a_0, a_1, a_2, \ldots, |a_{N+1}| + 1\}$, then $a_n < M$ for all $n \in \mathbb{N}$.
			Thus $(a_n)$ is bounded.
		\end{proof}
		
		\item 
		\begin{proof}
		Let $(a_n)$ be such a Cauchy sequence. From (ii), we know that $(a_n)$ is bounded. By Theorem
		6.6 
		\newtheorem*{tm1}{Theorem 6.6}
		\begin{tm1}
			Every bounded sequence of real numbers has a convergent susbequence.
		\end{tm1}
		We have, $(a_n)$ has a convergent subsequene, and since it converges, it is a Cauchy sequence
		as proved in (i).		
		\end{proof}
		
		\item
		\begin{proof} 
		Let $A = (a_n)$ be such a Cauchy sequence, then given $\epsilon > 0$, $\exists N \in \mathbb{N}$ 
		such that if $n, m > N$ then 
		$$|a_n - a_m| < \dfrac{\epsilon}{2}$$
		Also, let $B = (a_{n_i}) = \{a_{n_1}, a_{n_2}, a_{n_3}, a_{n_4}, \ldots\}$ 
		be a convergent subsequence of $A$ that converges to $L$, then $\exists
	 	M \in \mathbb{N}$ such that 
	 	$$|a_M - L| < \dfrac{\epsilon}{2}$$
	 	Hence, if $n \geq M$, we have
	 	\begin{eqnarray*}
	 		|a_n - L| &=& |a_n - a_M + a_M - L| \\
	 		& \leq & |a_n - a_M| + |a_M - L| \text{ (Triangle Inequality) }\\
	 		& < & \dfrac{\epsilon}{2} + \dfrac{\epsilon}{2} \\
	 		& = & \epsilon
	 	\end{eqnarray*}
	 	Therefore, $(a_n)$ converges.
		\end{proof}					
	\end{enumerate}
	
	\phantomsection
	\subsection*{{\color{purple}\underline{Problem 8}}}
	\addcontentsline{toc}{subsection}{\numberline{}Problem 8}	
	\begin{enumerate}
		\item Prove that a set $U \subset \mathbb{R}$ is open if and only if for any $x \in U$
		there is $r > 0$ such that the interval $(x - r, x + r) \subset U$.
		\item Prove that a set $U \subset \mathbb{R}$ is open if and only if $U$ is a union of open intervals.
		\item Prove that a set $U \subset \mathbb{R}$ if and only if $U$ is a union of countably many disjoint open
		intervals.
	\end{enumerate}
	
	
	\phantomsection
	\subsection*{{\color{purple}\underline{Problem 9}}}
	\addcontentsline{toc}{subsection}{\numberline{}Problem 9}
	Give an example of open sets $U_1 \supset U_2 \supset U_2 \ldots$ in $\mathbb{R}$ such that the intersection
	$\displaystyle\bigcap_{n=1}^{\infty} U_n$ is closed and non-empty.
	
	\phantomsection
	\subsection*{{\color{purple}\underline{Problem 10}}}
	\addcontentsline{toc}{subsection}{\numberline{}Problem 10}
	Give an example of closed sets $C_1 \supset C_2 \supset C_3 \ldots$ in $\mathbb{R}$ such that the intersection
	$\displaystyle\bigcup_{n=1}^{\infty} C_n$ is empty.
	
	\phantomsection
	\subsection*{{\color{purple}\underline{Problem 11}}}
	\addcontentsline{toc}{subsection}{\numberline{}Problem 11}
	\begin{enumerate}[(a)]
		\item Suppose that $\displaystyle\lim_{x\to\ a}f(x) = L \neq 0$. Prove that 
		$\displaystyle\lim_{x\to\ a}\dfrac{1}{f(x)} = \dfrac{1}{L}$.
	\begin{proof}
		By definition of limit, fix $\epsilon > 0$, there exists $\delta > 0$ such that
		$|x - a| < \delta \Rightarrow |f(x) - L| < \epsilon \cdot f(x) \cdot L$. Consider,
		$$\bigg|\dfrac{1}{f(x)} - \dfrac{1}{L}\bigg|
		= \bigg|\dfrac{L - f(x)}{f(x) \cdot L}\bigg| < \dfrac{\epsilon f(x) \cdot L}{f(x) \cdot L} = \epsilon$$
	\end{proof}
		\item Let $r(x) = \dfrac{p(x)}{q(x)}$ be a rational function, where $p(x)$ and $q(x)$ are
		polynomials in $x$. Prove that $r$ is continuous at all $x$ such that $q(x) \neq 0$.
	\begin{proof}
	
	\end{proof}
	\end{enumerate}
	
	\phantomsection
	\subsection*{{\color{purple}\underline{Problem 12}}}
	\addcontentsline{toc}{subsection}{\numberline{}Problem 12}
	Let $I$ be an interval in $\mathbb{R}$ and let $a \in I$. If $f$ is a function whose domain contains
	$I \setminus \{a\}$ define
	$$\displaystyle\lim_{x\to\ a+} f(x) = \displaystyle\lim_{x\to\ a} f_{+}(a)$$
	where $f_{+}$ is the function with domain $I \cap (a, \infty)$ given by $f_{+}(x) = f(x)$. Similarly, 
	define 
	$$\displaystyle\lim_{x\to\ a_{-}}f(x) = \displaystyle\lim_{x\to a}f_{-}(x)$$
	where $f_{-}$ is the function with domain $I \cap (-\infty, a)$ given by $f_{+}(x) = f(x)$. Prove that
	$\displaystyle\lim_{x\to\ a}f(x)$ exists if and only if both $\displaystyle\lim_{x\to\ a+}$ and
	$\displaystyle\lim_{x\to\ a-}f(x)$ exist and are equal.
	
	\phantomsection
	\subsection*{{\color{purple}\underline{Problem 13}}}
	\addcontentsline{toc}{subsection}{\numberline{}Problem 13}
	 Let $f$ be a real valued function defined on $(a, \infty)$, where $a > 0$ is some positive real number.
	 Let $\displaystyle\lim_{x\to\infty+}f(x)$ given by
	 $$\displaystyle\lim_{x\to\infty+} = \displaystyle\lim_{y\to\ 0}g(y)$$
	 where $g: (0, 1/a) \rightarrow \mathbb{R}$ is given by $g(y) = f(1/y)$ if the latter limit exists.
	 Prove that $\displaystyle\lim_{x\to\infty+}f(x)$ exists if and only if for any $\epsilon > 0$ there
	 exists a number $N \geq a$ such that $|f(x) - f(y)| < \epsilon$ if $x, y > N$.
	
	\phantomsection
	\subsection*{{\color{purple}\underline{Problem 14}}}
	\addcontentsline{toc}{subsection}{\numberline{}Problem 14}
	\begin{enumerate}[(a)]
		\item If $\displaystyle\lim_{x\to a}g(x)$ does not exist, can $\displaystyle\lim_{x\to a}[f(x) + g(x)]$
		exist? Can $\displaystyle\lim_{x\to a}f(x)g(x)$ exist?
		\begin{proof}
		Yes, for example if $g(x) = 1 - f(x)$ then $\displaystyle\lim_{x\to a}[f(x) + g(x)] = 
		\displaystyle\lim_{x\to a}[f(x) + 1 - f(x)]$ exists even if $\displaystyle\lim_{x\to a}f(x)$
		does not exist; and if $g(x) = \dfrac{1}{f(x)}$ where $f(x) \neq 0$ for all $x \neq a$, then
		$\displaystyle\lim_{x\to a}f(x)g(x)$ does exists even if $\displaystyle\lim_{x\to a}f(x)$
		and $\displaystyle\lim_{x\to a}g(x)$ do not exist. For example, if $f(x) = 1/(x - a)$ for $x \neq a$,
		and $g(x) = x - a$.
		\end{proof}
		
		\item If $\displaystyle\lim_{x\to a}f(x)$ exist and $\displaystyle\lim_{x\to a}[f(x) + g(x)]$ exists,
		must $\displaystyle\lim_{x\to a}g(x)$ exists?
		\begin{proof}
			Yes, again we can write $g(x) = (f(x) + g(x)) - f(x)$.Each of the terms $f(x) + g(x)$
			and $f(x)$ on the right side has limit when $x \rightarrow a$,
			so their difference also has limit when $x \rightarrow a$.
		\end{proof}
		
		\item If $\displaystyle\lim_{x\to a}f(x)$ exists, and $\displaystyle\lim_{x\to a}g(x)$ does not exist,
		can $\displaystyle\lim_{x\to a}[f(x) + g(x)]$ exist?
		\begin{proof}
			No. This is just another of starting part (b).
		\end{proof}
		
		\item If $\displaystyle\lim_{x\to a}f(x)$ exist, and $\displaystyle\lim_{x\to a}f(x)g(x)$ exists,
		does it follow that $\displaystyle\lim_{x\to a}g(x)$ exists?	
		\begin{proof}
			No. Let $f(x) = 0$ for all $x$ and let $g(x) = 1$ if $x$ is rational and $-1$ if $x$
			is irrational. Then for any $a$, $f(x)$ and $f(x)g(x) = 0$ both have limit when $x \rightarrow a$.
			but the limit of $g(x)$ does not exist when $x \rightarrow a$.
		\end{proof}
	\end{enumerate}
	
	\phantomsection
	\subsection*{{\color{purple}\underline{Problem 15}}}
	\addcontentsline{toc}{subsection}{\numberline{}Problem 15}
	\begin{enumerate}[(a)]
		\item Prove that if $0 < a < 2$ then $a < \sqrt{2a} < 2$.
		\begin{proof}
			If $0 < a < 2$ then $a^2 < 2a < 4 \Rightarrow a < \sqrt{2a} < 2$.
		\end{proof}
		\item Prove that the sequence,,
		$$\sqrt{2}, \sqrt{2\sqrt{2}}, \sqrt{2\sqrt{2\sqrt{2}}}, \ldots$$
		converges.
		\begin{proof}
			Part (a) show that
			$\sqrt{2} < \sqrt{2\sqrt{2}} < \sqrt{2\sqrt{2\sqrt{2}}} < \ldots < 2$.
			so by Monotonic Convergence Theorem, the sequence converges. 
		\end{proof}
		\item Find the limit. Hint: Notice that if $\displaystyle\lim_{n\to \infty}a_n = L$
		then $\displaystyle\lim_{n\to \infty}\sqrt{2a_n} = \sqrt{2L}$.
		\begin{proof}
			If this sequence is denoted by $\{a_n\}$, then the sequence $\{\sqrt{2a_n}\}$ is
			the same as $\{a_{n+1}\}$ for all $n$. So the hint show that $L = \sqrt{2L} \Leftrightarrow L = 2$.
		\end{proof}
	\end{enumerate}
	
	\phantomsection
	\subsection*{{\color{purple}\underline{Problem 16}}}
	\addcontentsline{toc}{subsection}{\numberline{}Problem 16}
	\begin{enumerate}[(a)]
		\item Prove that a convergent sequence is always bounded.
		\begin{proof}
			Suppose that $\displaystyle\lim_{n\to \infty}a_n = L$. Choose $N$ so that 
			$|a_n - L| < 1$ for $n > N$. Then $|a_n| < \max(|L| + 1, |a_1|, |a_2|, \ldots, |a_N|)$ for all $n$.
		\end{proof}
		\item Suppose that $\displaystyle\lim_{n\to \infty}a_n = 0$, and that some $a_n > 0$. Prove that
		the sets of all numbers $a_n$ actually has a maximum member.
		\begin{proof}
			Choose $N$ so that $|a_n - 0| < a_1$ for $n > N$. Then the maximum of $a_1, a_2, a_3, \ldots, a_N$
			is the maximum of $a_n$ for all $n$.
		\end{proof}
	\end{enumerate}
	
	\phantomsection
	\subsection*{{\color{purple}\underline{Problem 17}}}
	\addcontentsline{toc}{subsection}{\numberline{}Problem 17}
	Prove that $\displaystyle\lim_{n\to \infty}a_n = L$, then
	$$\displaystyle\lim_{n\to \infty} \dfrac{a_1 + \ldots + a_n}{n} = L$$
	\begin{proof}
		If $\epsilon > 0$, pick $N$ so that $|a_n - L| < \epsilon$. Then 
		$$|a_N + a_{N+1} + a_{N + 2} + a_{N + M} - ML| < \epsilon M$$
		so 
		$$\bigg|\dfrac{1}{N + M} \cdot [a_N + a_{N+1} + \ldots + a_{N+M}]  - \dfrac{ML}{N + M}\bigg| < \dfrac{\epsilon M}{N + M} < \epsilon$$
		Choose $M$ so that,
		$$\bigg|\dfrac{ML}{N + M} - L\bigg| < \epsilon \text{ and } \bigg| \dfrac{1}{N + M}[a_1 + a_2 + \ldots + a_N]\bigg| < \epsilon$$
		Then,
		$$\bigg| \dfrac{1}{N + M}[a_1 + a_2 + \ldots + a_N] - L\bigg| < 3 \epsilon$$
	\end{proof}
	
	\phantomsection
	\subsection*{{\color{purple}\underline{Problem 18}}}
	\addcontentsline{toc}{subsection}{\numberline{}Problem 18}
	\begin{enumerate}[(a)]
		\item Prove that if $\displaystyle\lim_{n\to \infty}(a_{n+1} - a_n) = L$, then $\displaystyle\lim_{n\to \infty}a_n/n = L$. 
		\begin{proof}
		Let $b_n = a_{n+1} - a_n$. Then $\{b_n\}$ satisfies the hypothesis of Problem 17 and the conclusion says that
		$$L = \displaystyle\lim_{n\to \infty} \dfrac{b_1 + \ldots + b_n}{n} = \displaystyle\lim_{n\to \infty}\dfrac{a_{n+1} - a_1}{n} = \displaystyle\lim_{n\to \infty}
		\dfrac{a_n}{n}$$
		
		\end{proof}
		\item Suppose that $f$ is continuous and $\displaystyle\lim_{x\to \infty} [f(x+1) - f(x)] = L$. Prove that
		$\displaystyle\lim_{x\to \infty}f(x)/x = L$.
		\begin{proof}
		The hypothesis $\displaystyle\lim_{x\to \infty}[f(x+1) - f(x)] = L$ implies that $a_n$ and $b_n$, the $\inf$ and
		$\sup$ of $f$ on $[n, n + 1]$ satisfy $\displaystyle\lim_{n\to \infty}[a_{n+1} - a_n] = L$ and 
		$\displaystyle\lim_{n\to \infty}[b_{n+1} - b_n] = L$. So by part (a), we have 
		$\displaystyle\lim_{n\to \infty}a_n/n = \displaystyle\lim_{n\to \infty}b_n/n = L$, which implies that
		$\displaystyle\lim_{x\to \infty}f(x)/x = L$.
		\end{proof}
	\end{enumerate}
	
	\phantomsection
	\subsection*{{\color{purple}\underline{Problem 19}}}
	\addcontentsline{toc}{subsection}{\numberline{}Problem 19}
	\begin{enumerate}
		\item Suppose that $\{a_n\}$ is a convergent sequence of points all in $[0, 1]$. Prove that
		$\displaystyle\lim_{n\to \infty} a_n$ is also in $[0, 1]$.
		\begin{proof}
			Suppose $\displaystyle\lim_{n\to \infty}a_n = L > 1$. Since $L - 1 > 0$, there would be some $n$
			with $|L - a_n| < L - 1$, hence $a_n > 1$, a contradiction. Similarly, we cannot have $L < 0$.
		\end{proof}
		
		\item Find a convergent sequence $\{a_n\}$ of points all in $(0, 1)$ such that 
		$\displaystyle\lim_{n\to \infty}a_n$ is not on $(0, 1)$.
		\begin{proof}
			$a_n = 1/n$.
		\end{proof}
	\end{enumerate}
	
	\phantomsection
	\subsection*{{\color{purple}\underline{Problem 20}}}
	\addcontentsline{toc}{subsection}{\numberline{}Problem 20}
	Suppose that $f$ is a function on $\mathbb{R}$ such that
	$$|f(x) - f(y)| \leq c|x - y| \text{     for all } x, y$$
	Prove that $f$ is continuous.
	\begin{proof}
		If $c = 0$ then $f$ is constant, so continuous. If $c \neq 0$, then $\epsilon > 0$, then 
		$|f(x) - f(a)| < \epsilon$ for $|x - a| < \epsilon/c$.
	\end{proof}
\end{document}
