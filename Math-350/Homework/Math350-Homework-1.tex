\documentclass[10pt,letterpaper]{article}
\renewcommand{\rmdefault}{ptm}

\usepackage[left=1in,right=1in,top=1in,bottom=1in]{geometry} 
\usepackage{amsmath}
\usepackage{amsfonts}
\usepackage{amsthm}
\usepackage{amssymb}
\usepackage{polynomial}
\usepackage{layouts}
\usepackage{enumerate}
\usepackage{syntax}
\usepackage{gensymb}
\usepackage{enumitem}
\usepackage{cancel}
\usepackage{calc}

\usepackage{xcolor}

\usepackage{minted}

\usepackage[version=0.96]{pgf}
\usepackage{tikz}
\usetikzlibrary{arrows,shapes,automata,backgrounds,petri,positioning}
\usetikzlibrary{decorations.pathmorphing}
\usetikzlibrary{decorations.shapes}
\usetikzlibrary{decorations.text}
\usetikzlibrary{decorations.fractals}
\usetikzlibrary{decorations.footprints}
\usetikzlibrary{shadows}
\usetikzlibrary{calc}
\usetikzlibrary{spy}
\usetikzlibrary{matrix}

\usepackage{tikz-qtree}

\setcounter{tocdepth}{2}
\setcounter{secnumdepth}{4}
\usepackage[bookmarksopen,bookmarksdepth=3]{hyperref}
\usepackage{titlesec}


%define new colors
\definecolor{dark-red}{rgb}{0.8,0.15,0.15}
\definecolor{dark-blue}{rgb}{0.15,0.15,0.7}
\definecolor{medium-blue}{rgb}{0,0,0.5}

%set up color for table of contents
\hypersetup{
    colorlinks, linkcolor={dark-red},
    citecolor={dark-blue}, urlcolor={medium-blue}
}

\usepackage{tocloft}

%preven linebreak between subsection header and its content
\titleformat{\subsection}[runin]{\normalfont\bfseries}{\thesubsection.}{2pt}{}
%\titleformat{\section}[runin]{\normalfont\bfseries\filcenter}{\thesection.}{5pt}{}


\titleformat{\section}[block]
{\normalfont\sffamily\LARGE}
{\thesection}{.2em}{\titlerule\\[.2ex]\bfseries}

%title
\title{\textbf{Math 350 - Advanced Calculus \\ Homework 1}}
\author{Chan Nguyen}

%set numwidth of section
\setlength{\cftsecnumwidth}{1.5cm} 
%make subsection numwidth different than as section
\setlength{\cftsubsecnumwidth}{3cm}
%make subsection indent the same as section
\setlength{\cftsubsecindent}{\cftsecindent} 

\newcommand{\sol}{\textbf{Solution. }}

\usepackage{tikz}
\usetikzlibrary{matrix}
\usetikzlibrary{shapes,backgrounds}

\makeatletter
\newcommand{\DESCRIPTION@original@item}{}
\let\DESCRIPTION@original@item\item
\newcommand*{\DESCRIPTION@envir}{DESCRIPTION}
\newlength{\DESCRIPTION@totalleftmargin}
\newlength{\DESCRIPTION@linewidth}
\newcommand{\DESCRIPTION@makelabel}[1]{\llap{#1}}%
\newcommand{\DESCRIPTION@item}[1][]{%
  \setlength{\@totalleftmargin}%
       {\DESCRIPTION@totalleftmargin+\widthof{\textbf{#1 }}-\leftmargin}%
  \setlength{\linewidth}
       {\DESCRIPTION@linewidth-\widthof{\textbf{#1 }}+\leftmargin}%
  \par\parshape \@ne \@totalleftmargin \linewidth
  \DESCRIPTION@original@item[\textbf{#1}]%
}
\newenvironment{DESCRIPTION}
  {\list{}{\setlength{\labelwidth}{0cm}%
           \let\makelabel\DESCRIPTION@makelabel}%
   \setlength{\DESCRIPTION@totalleftmargin}{\@totalleftmargin}%
   \setlength{\DESCRIPTION@linewidth}{\linewidth}%
   \renewcommand{\item}{\ifx\@currenvir\DESCRIPTION@envir
                           \expandafter\DESCRIPTION@item
                        \else
                           \expandafter\DESCRIPTION@original@item
                        \fi}}
  {\endlist}
\makeatother

\begin{document}

\tableofcontents 
\maketitle

\setlength{\parindent}{0pt}
\setlength{\parskip}{1ex}
	\phantomsection
	\subsection*{{\color{purple}\underline{Problem 1}}}
	\addcontentsline{toc}{subsection}{\numberline{}Problem 1}
	Let $X, Y$ be subsets of a set $S$, and let $C$ denote the complement with respect to $S$ that is
	$C(X) = S \setminus X$. Prove that: \\
	(a) $C(C(X)) = X$ \\
	(b) $C(X \cup Y) = C(X) \cap C(Y)$ \\
	(c) $C(X \cap Y) = C(X) \cup C(Y)$ \\
	(d) $X \cap Y = \emptyset \Leftrightarrow X \subset C(Y) \Leftrightarrow Y \subset C(X)$ \\
	(e) $X \cup Y = S \Leftrightarrow C(X) \subset Y \Leftrightarrow C(Y) \subset X$ \\
	
	\textbf{Solution.} \\
	(a) $C(C(X))= C(S \setminus X) = S \setminus (S \setminus X) = X$ \\
	Or: If $x \in X \implies x \not\in C(X) \implies x \in C(C(X) \implies C(C(X)) = X$ \\
	(b) $C(X \cup Y) = S \setminus (X \cup Y) = (S \setminus X) \cap (S \setminus Y) = C(X) \cap C(Y)$ \\
	Or: If $x \in C(X \cup Y) \implies x \in (C(X) \cap C(Y)) \implies x \not\in X$ and $x \not\in Y \implies x \in C(X)$ and 
	$x \in C(Y) \implies C(X \cup Y) = C(X) \cap C(Y)$ \\
	(c) $C(X \cap Y) = S \setminus (X \cap Y) = (S \setminus X) \cup (S \setminus Y) = C(X) \cup C(Y)$ \\
	Or: If $x \in C(X \cap Y) \implies x \in (C(X) \cup C(Y)) \implies x \in C(X)$ and $x \in C(Y) \implies
	C(X \cap Y) = C(X) \cup C(Y)$ \\
	(d) Without loss of generality, proving $X \subset C(Y)$ implies $Y \subset C(X)$ \\
	Assume $X$ and $Y$ in $S$, if $X \cap Y = \emptyset \implies C(Y) = (X \cup (S \setminus Y)) 
	\implies X \subset C(Y)$ \\
	(e) If $X \cup Y = S \implies C(X) = Y \cap (X \cap Y) \implies C(X) \subset Y$\\
	\phantomsection
	\subsection*{{\color{purple}\underline{Problem 2}}}
	\addcontentsline{toc}{subsection}{\numberline{}Problem 2}
	Let $F: X \rightarrow Y$ be a mapping of the set $X$ into the set $Y$. Let $A$ and $B$ be subsets of $X$, and let 
	$C$ and $D$ be subsets of $Y$. Prove or give a counterexample: \\
	(a) $F(A \cup B) = F(A) \cup F(B)$ \\
	(b) $F(A \cap B) = F(A) \cap F(B)$ \\
	(c) $F^{-1}(C \cup D) = F^{-1}(C) \cup F^{-1}(D)$ \\
	(d) $F^{-1}(C \cap D) = F^{-1}(C) \cap F^{-1}(D)$ \\
	(e) $F^{-1}(F(A)) = A$ \\
	(f) $F(F^{-1}(C)) = C$ \\
	
	\textbf{Solution.} \\
	(a) True. By definition of mapping we have:
	$$F(A) = \{ y \in Y \mid (\exists x \in A)(F(x) = y) \}$$
	$$F(B) = \{ y \in Y \mid (\exists x \in B)(F(x) = y) \}$$
	where $A \cup B = \{x \mid x \in A \cup B\} \implies F(A \cup B) = \{ y \in Y \mid (\exists x \in (A \cup B))(F(x) = y) \}$.
	Thus $F(A \cup B) = F(A) \cup F(B)$ \\
	(b) False. It's only true if $F$ is one to one. Counter example:
	$$F: y = x^2$$
	$A = \{2, -1\}, B = \{2, 1\}$
	Then $F(A \cap B) = \{4\}$ but $F(A) \cap F(B) = \{4, 1\}$ \\
	(c) True. Assume $C, D$ are subsets of $Y$ under the mapping $F: X \rightarrow Y$, then
	$$F^{-1}(C) = \{x \in X \mid F(x) \in C\}$$
	$$F^{-1}(D) = \{x \in X \mid F(x) \in D\}$$
	Then $F^{-1}(C \cap D) = F^{-1}(C) \cup F^{-1}(D)$
	(d) True. Assume $C, D$ are subsets of $Y$ under the mapping $F: X \rightarrow Y$, then
	$$F^{-1}(C) = \{x \in X \mid F(x) \in C\}$$
	$$F^{-1}(D) = \{x \in X \mid F(x) \in D\}$$
	Then $F^{-1}(C \cap D) = F^{-1}(C) \cap F^{-1}(D)$
	(e) False. Counterexample: $F : x \rightarrow x^2$, let $A = \{-1, 1\}$, then $F(-1) = F(1) = 1$, but
	$F^{-1}(A) = \{1\}$. In fact it's only true if $F$ is one to one\\
	(f) False. Counterexample: same as above with $C = \{1, -1\}$, then $F(F^{-1}(C)) = {1}$. Again, it's only
	true if $F$ is one to one
	
	\phantomsection
	\subsection*{{\color{purple}\underline{Problem 3}}}
	\addcontentsline{toc}{subsection}{\numberline{}Problem 3}
	Let $F: X \rightarrow Y$ be a mapping of the set $X$ into the set $Y$. Prove the following properties are equivalent: \\
	(a) $F$ is injective (one to one) \\
	(b) For any subset $A$ of $X$, $F^{-1}(F(A)) = A$ \\
	(c) For any pair of subsets $A, B$ of $X$, $F(A \cap B) = F(A) \cap F(B)$ \\
	(d) For any pair of subsets $A, B$ of $X$ such that $A \cap B = \emptyset$, the intersection $F(A) \cap F(B) = \emptyset$ \\
	(e) For any pair of subsets $A, B$ of $X$ such that $B \subset A$, the image $F(A \setminus B) = F(A) \setminus F(B)$ \\
	
	\textbf{Solution.} \\
	(a) Use (b) to prove (a): \\
		For any $A \subset X$, if $F^{-1}(F(A))$ then $F$ must be one to one.
	(b) Use (a) to prove (b): \\ 
		Because $F$ is one to one, $F(A)$ is a subset of $Y$, then $F^{-1}(F(A)) = A$\\
	(c) Use (c) to prove (a): \\ 
	We have:
	\begin{itemize}
		\item $F(A) = \{ y \in Y \mid (\exists x \in A)(F(x) = y)\}$
		\item $F(B) = \{ y \in Y \mid (\exists x \in B)(F(x) = y)\}$
	\end{itemize}
 	where $F(A \cap B) = \{ y \in Y \mid (\exists x \in (A \cap B))(F(x) = y)\}$ which implies $F$ must be one to one \\
 	(d) Use (d) to prove (a): \\
 	If $A \cap B = \emptyset$ implies $F(A) \cap F(B) = \emptyset$ is only true if $F$ is one to one otherwise $F(A) \cap
 	F(B) \neq \emptyset$ \\
 	(e) Use (a) to prove (e): \\
 	Again this is only true if $F$ is one to one, because $A \setminus B$ implies $x \in A$ and $x \not\in B$, which could
 	be decomposed into $F(A) \setminus F(B)$. \\
 	
	\phantomsection
	\subsection*{{\color{purple}\underline{Problem 4}}}
	\addcontentsline{toc}{subsection}{\numberline{}Problem 4} \text{  } \\
	(a) How many subsets are there of the set $\{1, 2, 3, \ldots, n\}$? \\
	(b) How many functions from this set to itself? \\
	(c) How many injective (one to one) mappings of this set into itself? \\
	(d) How many surjective (onto) mappings of this set into itself? \\
	
	\textbf{Solution.} \\
	(a) $2^{n}$ \\
	(b) $n$  \\
	(c) $n!$ \\
	(d) $n!$ \\
	
	\phantomsection
	\subsection*{{\color{purple}\underline{Problem 5}}}
	\addcontentsline{toc}{subsection}{\numberline{}Problem 5} Let $\mathbf{Z_{+}}$ denote the set of positive
	integers $\mathbf{Z_{+}} = \{1, 2, 3, \ldots \}$. Let $F: \mathbf{Z_{+}} \times \mathbf{Z_{+}} \rightarrow \mathbf{Z_{+}}$
	be the mapping given by:
	$$F(x, y) = \dfrac{(x + y - 2)(x + y - 1)}{2} + y$$
	Prove that $F$ is bijection (one to one and onto).\\
	
	\textbf{Solution.} \\
	\begin{proof} To prove $F$ is bijection, we need to prove it's injective and surjective. \\
	\begin{itemize}
		\item one-to-one \\
		Suppose there exists $x', y'$ where $x \neq x', y \neq y'$ such that $F((x, y)) = F((x',y')$:
		$$\Leftrightarrow \dfrac{(x + y - 2)(x + y - 1)}{2} + y = \dfrac{(x' + y' - 2)(x' + y' - 1)}{2} + y'$$
		However, this is impossible because the extra $y$ at the end. We can have multiple pairs of $(x, y)$ such that:
		$x + y = x_0 + y_0 = x_1 + y_1 \ldots$ but we can only have one $y$. Thus $x = x'$ and $y = y'$. Therefore
		$F$ is one to one. (1)
		\item onto \\
		What we need to show is that for any $z \in \mathbf{Z_{+}}$, there exits a pair solution $(x, y)$. \\
		If we rewrite $F$ as:
		$$F(x, y) = \dfrac{(x + y - 2)(x + y - 2 + 1)}{2} + y$$ 
		we can see that it's of the form accumulative sum $f(n) = \dfrac{n(n + 1)}{2}$, plug in some $n$, we see that:
		$$f(1) = \dfrac{1(1 + 1)}{2} = 1$$
		$$f(2) = \dfrac{2(2 + 1)}{2} = 3$$
		$$f(3) = \dfrac{3(3 + 1)}{2} = 6$$
		$$f(4) = \dfrac{4(4 + 1)}{2} = 10$$
		$$ \ldots $$	
		However, the results do not consists of all positive integers in $\mathbf{Z_{+}}$, there are gaps between
		$1 \rightarrow 3 \rightarrow 6 \rightarrow 10 \ldots$. Fortunately, $n = x + y$ and the extra $y$ at the end
		of this mapping indeed generate all these positive numbers. Also the extra $-2$ in the expression does yield
		a bigger range for $x$ and $y$. Assume we want $n = 3$, there are two pairs of $(x, y)$ which yields the same
		result in $f(n) = \dfrac{n(n + 1)}{2}$, where the extra $y$ at the end could make the difference. \\
		If we choose $y = 1$, then the result is:
		$$F(2, 1) = \dfrac{(3 - 2)(3 - 1)}{2} + 1 = 1 + 1 = 2$$
		If we choose $y = 2$, then the result is:
		$$F(2, 1) = \dfrac{(3 - 2)(3 - 1)}{2} + 2 = 1 + 2 = 3$$
		Similarly for $n = 4, 5, 6, \ldots$, and as $n$ becomes larger and larger we have more solution
		to the Diophantine equations $x + y = n$, for $x, y \geq 1$. In fact, the number of solutions of
		non-negative integer values to the equation $n = x + y$ is:
			$$\dbinom{n + 2 - 1}{2 - 1} = (n + 1)$$
		So it's sufficient to generate all positive integers in $\mathbf{Z_{+}}$
		Note that although we don't have $x, y \geq 1$, $x + y - 1$ and $x + y - 2$ does include
		these two cases. The other way to look at it is to subtract the two cases $(0, n)$ and $(n, 0)$
		which still yields $n + 1 - 2 = n - 2$ choices for $x, y$.
		Thus this mapping yields all positive integers in $\mathbf{Z_{+}}$ which implies
		$F$ is onto. (2)
	\end{itemize}
	From (1) and (2) we can conclude that $F$ is bijection.
	\end{proof}
	
\end{document}
