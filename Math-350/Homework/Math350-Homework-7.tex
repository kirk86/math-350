\documentclass[10pt,letterpaper]{article}
\renewcommand{\rmdefault}{ptm}

\usepackage[left=1in,right=1in,top=1in,bottom=1in]{geometry} 
\usepackage{amsmath}
\usepackage{amsfonts}
\usepackage{amsthm}
\usepackage{amssymb}
\usepackage{polynomial}
\usepackage{layouts}

\usepackage{enumerate}

\usepackage{syntax}
\usepackage{gensymb}
\usepackage{cancel}
\usepackage{calc}
\usepackage{xcolor}

\usepackage{minted}

\usepackage[version=0.96]{pgf}
\usepackage{tikz}
\usetikzlibrary{arrows,shapes,automata,backgrounds,petri,positioning}
\usetikzlibrary{decorations.pathmorphing}
\usetikzlibrary{decorations.shapes}
\usetikzlibrary{decorations.text}
\usetikzlibrary{decorations.fractals}
\usetikzlibrary{decorations.footprints}
\usetikzlibrary{shadows}
\usetikzlibrary{calc}
\usetikzlibrary{spy}
\usetikzlibrary{matrix}

\usepackage{tikz-qtree}

\setcounter{tocdepth}{2}
\setcounter{secnumdepth}{4}
\usepackage[bookmarksopen,bookmarksdepth=3]{hyperref}
\usepackage{titlesec}


%define new colors
\definecolor{dark-red}{rgb}{0.8,0.15,0.15}
\definecolor{dark-blue}{rgb}{0.15,0.15,0.7}
\definecolor{medium-blue}{rgb}{0,0,0.5}
\definecolor{dark-green}{rgb}{0.2,0.7,0.7}

%set up color for table of contents
\hypersetup{
    colorlinks, linkcolor={dark-green},
    citecolor={dark-blue}, urlcolor={medium-blue}
}

\usepackage{tocloft}

%preven linebreak between subsection header and its content
\titleformat{\subsection}[runin]{\normalfont\bfseries}{\thesubsection.}{2pt}{}
%\titleformat{\section}[runin]{\normalfont\bfseries\filcenter}{\thesection.}{5pt}{}


\titleformat{\section}[block]
{\normalfont\sffamily\LARGE}
{\thesection}{.2em}{\titlerule\\[.2ex]\bfseries}

%title
\title{\textbf{Math 350 - Advanced Calculus \\ Homework 7}}
\author{Chan Nguyen}

%set numwidth of section
\setlength{\cftsecnumwidth}{1.5cm} 
%make subsection numwidth different than as section
\setlength{\cftsubsecnumwidth}{3cm}
%make subsection indent the same as section
\setlength{\cftsubsecindent}{\cftsecindent} 

\newcommand{\sol}{\textbf{Solution}}

\usepackage{tikz}
\usetikzlibrary{matrix}
\usetikzlibrary{shapes,backgrounds}

\makeatletter
\newcommand{\DESCRIPTION@original@item}{}
\let\DESCRIPTION@original@item\item
\newcommand*{\DESCRIPTION@envir}{DESCRIPTION}
\newlength{\DESCRIPTION@totalleftmargin}
\newlength{\DESCRIPTION@linewidth}
\newcommand{\DESCRIPTION@makelabel}[1]{\llap{#1}}%
\newcommand{\DESCRIPTION@item}[1][]{%
  \setlength{\@totalleftmargin}%
       {\DESCRIPTION@totalleftmargin+\widthof{\textbf{#1 }}-\leftmargin}%
  \setlength{\linewidth}
       {\DESCRIPTION@linewidth-\widthof{\textbf{#1 }}+\leftmargin}%
  \par\parshape \@ne \@totalleftmargin \linewidth
  \DESCRIPTION@original@item[\textbf{#1}]%
}
\newenvironment{DESCRIPTION}
  {\list{}{\setlength{\labelwidth}{0cm}%
           \let\makelabel\DESCRIPTION@makelabel}%
   \setlength{\DESCRIPTION@totalleftmargin}{\@totalleftmargin}%
   \setlength{\DESCRIPTION@linewidth}{\linewidth}%
   \renewcommand{\item}{\ifx\@currenvir\DESCRIPTION@envir
                           \expandafter\DESCRIPTION@item
                        \else
                           \expandafter\DESCRIPTION@original@item
                        \fi}}
  {\endlist}
\makeatother

\begin{document}

\tableofcontents 
\maketitle

\setlength{\parindent}{0pt}
\setlength{\parskip}{1ex}
	\phantomsection
	\subsection*{{\color{purple}\underline{Problem 1}}}
	\addcontentsline{toc}{subsection}{\numberline{}Problem 1}
	\begin{enumerate}[(a)]	
	\item Suppose that f is a function satisfying $|f(x)| \leq |x|$ for all $x$. Show that $f$ is continuous
at $0$. \\
	\begin{proof}
		Recall definition of continuity, if $\displaystyle\lim_{x\to a}f(x) = f(a)$, then $f(x)$
		is continuous at $a$. Using the definition of limit, given a $\epsilon > 0$, we need
		to find $\delta = \delta(\epsilon)$ such that if $0 < |x - a| < \delta$ then $|f(x) - f(a)| < \epsilon$.
		Since $|f(x)| \leq |x| \Rightarrow 0 \leq |f(0)| \leq 0$, thus $f(0) = 0$. Fix
		$\epsilon > 0$, pick $\delta = \epsilon$ then $|x - 0| < \delta 
		\Leftrightarrow |x| < \epsilon \Rightarrow
		|f(x) - 0| = f(x) \leq |x| < \epsilon$ for all $x \in dom(f)$. Since $\epsilon$ is arbitrary,
		$\displaystyle\lim_{x\to 0}f(x) = f(0)$ or $f(x)$ is continuous at $0$.
	\end{proof}
	
	\item Give an example of such a function $f$ which is not continuous at any $a = 0$. 
	\begin{proof}
	An example of $f(x)$ not continuous at any $a \neq 0$ could be:
	$$f(x) = 
	\begin{cases}
		0  &, x \text{ rational } \\
		x  &, x \text{ irrational } 
	\end{cases}
	$$ 
	As we can see the limit $\displaystyle\lim_{x\to\ 0}f(x) = f(0)$ exists. But
	if $a \neq 0$, given $\epsilon > 0$, we
	can't find $\delta > 0$ such that if $0 < |x - 0| < \delta$ then $|f(x) - L| < \epsilon$
	no matter what $L$ is because by the "Denseness of $\mathbb{Q}$", if $a, b \in \mathbb{R}$,
	then there is a rational $r \in \mathbb{Q}$ such that $a < r < b$. So as we come close 
	to any $a \neq 0$, the limit alternates between $x$ and $0$. Thus it is only continuous at $0$.	
	\end{proof}
	
	\item Suppose that $g$ is continuous at $0, g(0) = 0$, and $f$ is a function satisfying 
	$|f(x)| \leq |g(x)|$ for all $x$. Show that $f$ is continuous at $0$. 
	\begin{proof}
		Since $g(x)$ is continuous at $0$ and $g(0) = 0$, we have
		$$\displaystyle\lim_{x\to\ 0}g(x) = g(0) = 0$$
		which can be translated to the definition of limit as,
		$\forall \epsilon > 0, \exists \delta > 0$, such that for all $x$ if $0 < |x - 0| = |x| < \delta$
		then $|g(x) - 0| = g(x) < \epsilon$. But $|f(x)| \leq |g(x)|$ for all $x$, thus we can choose 
		a $\delta > 0$, such that if $|x - 0| < \delta$ then $|g(x) - 0| = g(x) \leq |f(x)| < \epsilon$.
		In other words, $f(x)$ is continuous at $0$. 
	\end{proof}
	\end{enumerate}
	
	\newpage
	\phantomsection
	\subsection*{{\color{purple}\underline{Problem 2}}}
	\addcontentsline{toc}{subsection}{\numberline{}Problem 2}
	\begin{enumerate}[(a)]
	\item Prove that if $f$ is continuous at $a$, then so $|f|$.
	\begin{proof}
		Since $f$ is continuous at $a$, we have that:
		$$\displaystyle\lim_{x\to\ a}f(x) = f(a)$$
		By definition of limit, $\forall \epsilon > 0, \exists \delta > 0$ such that $\forall x$,
		if $0 < |x - a| < \delta$ implies $|f(x) - f(a)| < \epsilon$. On the other hand,
		using Triangle inequality, we have:
		$$|f(x)| - |f(a)| \leq |f(x) - f(a)| \text{ and } |f(x)| - |f(a)| \leq |f(a) - f(x)|$$
		which implies $||f(x)| - |f(a)|| \leq |f(x) - f(a)| < \epsilon$. Thus we can choose
		$\delta > 0$ such that if $0 < |x - a| < \delta$ then $||f(x)| - |f(a)|| < \epsilon$
		which is equivalent to $\displaystyle\lim_{x\to\ a}|f(x)| = |f(a)|$. Hence,
		$|f|$ is continuous at $a$. 
	\end{proof}
	\item Give an example of a function $f$ such that $f$ is continuous nowhere, but $|f|$
	is continuous everywhere.
	\begin{proof}
	We have to be careful here because it's easy to come up with an example where $f$ is
	continuous everywhere but $f$ is not continuous at some $a$, for example
	$$
	f(x) = 
	\begin{cases}
		1  &, x \geq 0  \\
		-1 &, x < 0
	\end{cases}
	$$
	To make $f$ is not continuous "nowhere" we have to use "rational" and 
	"irrational" trick as mentioned in class.
	$$
	f(x) = 
	\begin{cases}
		a  &, x \text{ rational } \\
		-a &, x \text{ irrational }
	\end{cases}
	$$
	for some $a \in \mathbb{N}$.
	As we can see $\displaystyle\lim_{x\to\ a}f(x) = \text{DNE}$ but $\displaystyle\lim_{x\to\ a}|f|(x) = a$.
	Hence $|f|$ is continuous at any $a$, while $f$ always alternates between $a$ and $-a$ no matter
	what $a$ is, thus $f$ is not continuous nowhere. 	
	
	\end{proof} 
	\item Prove that if $f$ and $g$ are continuous, so are $\max(f, g)$ and $\min(f, g)$.
	\begin{proof}
		Recall from Problem 4 - Homework 3, we have:
		$$\max(a, b) = \dfrac{a + b + |a - b|}{2}$$
		$$\min(a, b) = \dfrac{a + b - |a - b|}{2}$$
		Although we did not prove the $\min(a, b)$, we can prove it quite easily. 
		Consider three cases, \\
		If $a > b$, then 
			$$\min(a, b) = \dfrac{a + b - a + b}{2} = \dfrac{2b}{2} = b$$
		If $a < b$, then
			$$\min(a, b) = \dfrac{a + b - b + a}{2} = \dfrac{2a}{2} = a$$
		If $a = b$, then 
		 	$$\min(a, b) = \dfrac{a + b - 0}{2} = a = b$$  		
		Hence,
		We have,
			$$\max(f, g) = \dfrac{f + g + |f - g|}{2} \text{ and } \min(f, g) = \dfrac{f + g - |f - g|}{2}$$
		Apply Problem 3 (Algebra of Continuity) we have that $(f + g)$ and $(f - g)$ are continuous, where 
		apply part (a), we know that $|f - g|$ is also continuous. Thus $\max(f, g)$ and $\min(f, g)$ are
		continuous.
	\end{proof}		
	
	
	\end{enumerate}
	
	\phantomsection
	\subsection*{{\color{purple}\underline{Problem 3}}}
	\addcontentsline{toc}{subsection}{\numberline{}Problem 3}
	\begin{enumerate}[(a)]
	\item Prove that if $f$ and $g$ are continuous, then so is $f + g$.
	\begin{proof}
		Since $f(x), g(x)$ are continuous at $a$, we have that
		$$\displaystyle\lim_{x\to\ a}f(x) = f(a)$$
		$$\displaystyle\lim_{x\to\ a}g(x) = g(a)$$
		Apply Theorem 8.4 from lecture notes to $f$ and $g$, we have that
		$$\displaystyle\lim_{x\to\ a}(f + g)(x) = 
		\displaystyle\lim_{x\to\ a}f(x) + \displaystyle\lim_{x\to\ a}g(x)
		= f(a) + g(a) = (f + g)(a)$$
		Thus $f + g$ is continuous at $a$.
	\end{proof}
	
	\item Prove that if $f$ and $g$ are continuous, then so is $f \cdot g$.
	\begin{proof}
		Again, we can use Theorem 8.4 from lecture notes to obtain:
		$$\displaystyle\lim_{x\to\ a}(f \cdot g)(x) = 
		\displaystyle\lim_{x\to\ a}f(x) \cdot \displaystyle\lim_{x\to\ a}g(x)
		= f(a) \cdot g(a) = (f \cdot g)(a)$$
		Thus $f \cdot g$ is continuous at $a$.
	\end{proof}
	
	\item Give an example of two functions $f$ and $g$, both discontinuous at 0, whose sum is continuous at 0.
	
	
	\item Give an example of two functions $f$ and $g$, both discontinuous at 0, whose product is continuous at 0.
	\end{enumerate}
	
	\phantomsection
	\subsection*{{\color{purple}\underline{Problem 4}}}
	\addcontentsline{toc}{subsection}{\numberline{}Problem 4}
	Prove that 
	\begin{enumerate}[(a)]
	\item A function, $f$, is continuous if and only if $f$ takes convergent sequences to convergent sequences (that
is, if $(x_n)$ is a sequence in $\mathrm{dom}(f)$ that converges to a point in 
$\mathrm{dom}(f)$, then $(f(x_n))$ converges).
\begin{proof}
	Generally, \\
	\textbf{Theorem .} Let $f$ be a function with domain $\mathrm{dom}(f)$ and suppose $a \in \mathrm{dom}(f)$. Then 
	$f$ is continuous at $a$ if and only if whenever $\{x_n\}$ is a sequence in $\mathrm{dom}(f)$ which converges
	to $a$, then the sequence $\{f(x_n)\}$ converges to $f(a)$. \\
	We first prove the "only if" - that is we assume $f$ is continuous and proceed to prove the statement about the sequences.
	Let $\{x_n\}$ be a sequence in $\mathrm{dom}(f)$ with $x_n \rightarrow a$. Given $\epsilon > 0$, there is a $\delta > 0$ such
	that
	$$|f(x) - f(a)| < \epsilon \text{ whenever } x \in \mathrm{dom}(f) \text{ and } |x - a| < \delta $$
	For this $\delta$, there is an $N$ such that
	$$|x_n - a| < \delta \text{ whenever } n > N$$
	Combining these statements, we conclude that
	$$|f(x_n) - f(a)| < \epsilon \text{ whenever } n > N$$
	Thus $f(x_n) \rightarrow f(a)$. This completes the proof of "only if". \\
	
	We will prove the "if" part, by proving the contrapositive - that is, we will prove that if 
	$f$ is not continuous at $a$, there is a sequence $\{x_n\}$ in $\mathrm{dom}(f)$ such that 
	$x_n \rightarrow a$ but $\{f(x_n)\}$ does not converges to $f(a)$. \\
	The assumption that $f$ is not continuous at $a$ means that there is an $\epsilon > 0$ for 
	which no $\delta$ can be found for which the definition of continuous is true. This means
	that, no matter what $\delta$ we choose, there is always an $x \in \mathrm{dom}(f)$ such that
	$$|x - a| < \delta \text{ but } |f(x) - f(a)| \geq e$$
	In particular, for each of the number $\dfrac{1}{n}$ for $n \in \mathbb{N}$ we may choose
	$x_n \in \mathrm{dom}(f)$ such that 
	$$|x_n - a| < \dfrac{1}{n} \text{ but } |f(x_n) - f(a)| \geq \epsilon$$
	These numbers form a sequence $\{x_n\}$ which converges to $a$ since $\dfrac{1}{n} \rightarrow 0$.
	but the image sequence $\{f(x_n)\}$ does not converges to $f(a)$. This completes the proof.
	
	
\end{proof}


	\item A function, $f$, is continuous if and only if $f$ takes Cauchy sequences to Cauchy sequences (that is, if
$(x_n)$ is a Cauchy sequence in $\mathrm{dom}(f)$, then $(f(x_n))$ is a Cauchy sequence.	
	\begin{proof}
	See previous Homework.
	\end{proof}
	\end{enumerate}
	
	\phantomsection
	\subsection*{{\color{purple}\underline{Problem 5}}}
	\addcontentsline{toc}{subsection}{\numberline{}Problem 5} 
	Let I, J be open intervals in R and let $f : I \rightarrow J$ be a function that is increasing (that is, if
	$x < y$, then $f(x) < f(y)$) and surjective. Prove that $f$ is continuous.
	
	\phantomsection
	\subsection*{{\color{purple}\underline{Problem 6}}}
	\addcontentsline{toc}{subsection}{\numberline{}Problem 6} 
	\begin{enumerate}	
	\item Give an example of a family of open sets, $\{U_n\}_{n \in \mathbb{N}}$, of real numbers, 
	such that $U_1 \supset U_2 \supset U_3 \supset \ldots$ such that the intersection
	$\displaystyle\bigcap_{n \in \mathbb{N}}$ is closed and non-empty.
	\item 
	Give an example of a collection of closed sets $\{F_n\}_{n \in \mathbb{N}}$ of numbers, such that 
	$F_1 \supset F_2 \supset F_3 \supset \ldots$ such that
	$F_n \neq \emptyset$ for all $n$ and the intersection $\displaystyle\bigcap_{n \in \mathbb{N}} = \emptyset$
	\end{enumerate}
	
	\phantomsection
	\subsection*{{\color{purple}\underline{Problem 7}}}
	\addcontentsline{toc}{subsection}{\numberline{}Problem 7}
	\begin{enumerate}[(a)]
	\item Prove than any intersection of compact sets is a compact set.
	\begin{proof}
		Recall, \\
		\textbf{Theorem. }
		A set $S \subset \mathbb{R}$ is compact if only if it's closed and bounded. \\
		
		Let $S, T$ be compact sets, so $S, T$ are closed and bounded. Let $U = S \cap T$.
		If $U = \emptyset$, then $U$ is compact because $\emptyset$ is closed and bounded.
		If $U \neq \emptyset$, then $U$ is closed by Theorem 7.2 (iv). Hence, it remains to show that $U$ is bounded.
		Indeed, since $S, T$ are bounded, $S, T$ are bounded above and below, so 
		$\inf(S), \inf(T), \sup(S), \sup(T)$ exist. Let $m = \min(\inf(S), \inf(T))$ and 
		$s = \max(\sup(S), \sup(T))$ then for all $x \in U, x \geq m, x \leq s$, thus $U$ is bounded.
		Therefore, $U$ is compact.			
	\end{proof}
	
	\item Prove that a union of finitely many compact sets is a compact set.	
	\begin{proof}
		From Theorem 7.2 (iii), we have that \\
		\textbf{Theorem. } The union of finitely many closed sets is a closed set. 
		Hence, it remains to show that
		$U = \displaystyle\bigcup_{i=1}^{n} S_i$ is bounded. 
		Since there are only "finitely many" such sets, we can define
		$m = \min(\inf(S_1), \inf(S_2), \inf(S_3), \ldots, \inf(S_n))$ and 
		$s = \max(\sup(S_1), \sup(S_2), \sup(S_3), \ldots, \sup(S_n))$, then for all 
		$x \in U$, $x \geq m, x \leq s$, thus $U$ is bounded. Thus $U$ is compact.
	\end{proof}
	
	
	\end{enumerate}
	
	\phantomsection
	\subsection*{{\color{purple}\underline{Problem 8}}}
	\addcontentsline{toc}{subsection}{\numberline{}Problem 8}
	Prove or give a counterexample:
	\begin{enumerate}[(a)]
	\item Any union of infinitely many compact sets is compact.
	\begin{proof}
	False! Since compact implies closed and bounded, which implies
	union of infinitely many closed sets is closed which is clearly false.
	A counterexample could be:
		$$S = \displaystyle\bigcup_{i = 1}^{\infty}[-i, i]$$
	Each closed interval $[-i, i]$ is closed and bounded so it's compact but
	$S$ is not because $S$ is not bounded above.
		
	\end{proof}
	\item A non-empty subset $S \subset \mathbb{R}$ that has both a maximum and a minimum is compact.
	\begin{proof}
		False! a counter example could be:
		$$S = [-a, 0) \cup (0, a]$$
		S has $a$ as maximum and $-a$ as minimum but it's not closed because we can find a
		sequence $(a_n) = \dfrac{1}{n}$ which converges to $0$ but $0$ is not in $S$.
	\end{proof}		
	
	
	
	\end{enumerate}
	
	\phantomsection
	\subsection*{{\color{purple}\underline{Problem 9}}}
	\addcontentsline{toc}{subsection}{\numberline{}Problem 9}
	\begin{enumerate}[(a)]
	 \item Consider a family of compact intervals, $I_1 = [a_1, b_1], I_2 = [a_2, b_2], \ldots$. 
	 Suppose that $a_n \leq a_{n+1}$ and $b_{n+1} \leq b_n$ for all $n$. 
	 Prove that there is an $x$ which is in every $I_n$, that is, there is an 
	 $x \in \displaystyle\bigcap_{n=1}^{\infty} I_n$.
	 \begin{proof}
	 Visually, \\
	 	\begin{tikzpicture}
			\draw (0, 0) -- (14, 0);
			\draw (1, 0) node {$|$};
			\draw (1, 0) node[yshift=6mm] {$a_n$};
			\draw (13, 0) node {$|$};
			\draw (13, 0) node[yshift=6mm] {$b_n$};
			
			\draw (0, 1) -- (14, 1);
			\draw (3, 1) node {$|$};
			\draw (3, 1) node[yshift=6mm] {$a_{n+1}$};
			\draw (11, 1) node {$|$};
			\draw (11, 1) node[yshift=6mm] {$b_{n+1}$};
			
			\draw (0, 2) -- (14, 2);
			\draw (5, 2) node {$|$};
			\draw (5, 2) node[yshift=6mm] {$a_{n+2}$};
			\draw (9, 2) node {$|$};
			\draw (9, 2) node[yshift=6mm] {$b_{n+2}$};
			
			\draw (0, 3) -- (14, 3);
			\draw (6, 3) node {$|$};
			\draw (6, 3) node[yshift=6mm] {$a_{n+3}$};
			\draw (8, 3) node {$|$};
			\draw (8, 3) node[yshift=6mm] {$b_{n+3}$};
			
			\draw (7, 4) node {$\ldots$};	
		\end{tikzpicture} \\
		As we can see, it $x$ belongs to the smallest interval, it will be in all other intervals.
		To prove it formally, note that if $x \in [a_{n+1}, b_{n+1}]$ then $x \in [a_n, b_n]$ because
		$a_n \leq a_{n+1}$ and $b_{n+1} \leq b_{n}$ for all $n$. Thus by induction on $n$, $x$
		must be in all other intervals. In fact, another way to look at this problem is consider the 
		two sets:
		$$A = \{a_n: n \in \mathbb{N} \} \text{ and } B = \{b_n: n \in \mathbb{N}\}$$
		Since $A \neq \emptyset$ and $a_n \leq b_1$ for all $n \Rightarrow \sup(A)$ exists.
		Also, since $B \neq \emptyset$ and $b_1 \leq b_n$ for all $n \geq 2 \Rightarrow \inf(B)$ exists. 
		On the other hand, we have that
			$$a_n \leq a_{n+1} \text{ and } b_{n+1} \leq b_{n} ,\, \, \,  \forall n$$
		And this tell us that $\inf(B)$ is also an upper bound of $A$ and it's the least at 
		such, where $\sup(A)$ is also the a lower bound of $B$ and it's the greatest as such.
		In other words, we have to sequences, $(a_n)$ is non-decreasing, bounded above
		and converges to $\sup(A)$ whereas $(b_n)$ is non-increasing, bounded below and converges to $\inf(B)$. 
		Besides, since $a_n \leq a_{n+1} \leq a_{n+2} \leq \ldots \text{ and } 
		\ldots \leq b_{n+2} \leq b_{n+1} \leq b_{n} \Leftrightarrow
		a_n \leq a_{n + k} \leq b_{n + k} \leq b_n$ for $k \in \mathbb{N}$, we must have
		$\sup(A) \leq \inf(B)$. Hence if $a_n \leq x \leq b_n$ for 
		all $n$. Therefore $x \in I_n$ for all $n$ or there exists $x$ such that
		$x \in \displaystyle\bigcap_{n=1}^{\infty} I_n$.
	 \end{proof}
	 
	\item Prove that if the sequence length $I_n = b_n - a_n \neq 0$, then the $x$ in $(a)$ is unique. \\
	\begin{proof}
		From part (a), we have shown that $b_n \rightarrow \inf(B)$ and $(a_n) \rightarrow \sup(A)$.
		Using Algebra Limit Theorem, we have
		$$\displaystyle\lim_{n\to\infty}I_n = 
		\displaystyle\lim_{n\to\infty} (b_n - a_n) = 
		\displaystyle\lim_{n\to\infty} b_n - \displaystyle\lim_{n\to\infty}a_n = 
		\inf(B) - \sup(A) = 0$$
		Hence, $\inf(B) = \sup(A)$ so if $x \in \displaystyle\bigcap_{n=1}^{\infty} I_n$ then 
		$\sup(A) \leq x \leq \inf(B) \Rightarrow x = \inf(B) = \sup(A)$. Thus by the uniqueness property
		of $\inf$ and $\sup$, $x$ is unique. 
	\end{proof}
	
	\item Show that the conclusion in $(a)$ is false if we consider open intervals instead of closed intervals. 
	Is it true if we consider open and bounded intervals?
	\begin{proof}
	To show that $(a)$ is false, we need a counterexample interval $I_n$ such that 
	$x \in \displaystyle\bigcap_{n=1}^{\infty} I_n$ is false. In other words,
	if the intersection $\displaystyle\bigcap_{n=1}^{\infty} I_n = \emptyset$ then
	there is no such $x$. Hence if $(a_n) \rightarrow a$ (strictly increasing) and $(b_n) \rightarrow b$ (strictly
	decreasing), then an interval of the form 
			$$(a, b_n) \text{ and } (a_n, b)$$
	will do. Thus a counterexample would be
	$$I_n = \bigg(\dfrac{-1}{n}, 0\bigg) \text{ or } I_n = \bigg(0, \dfrac{1}{n}\bigg)$$
	As we can see, these intervals are both bounded by 0, so it's false even if we consider open
	and bounded intervals.
	\end{proof}
	
	\end{enumerate}
	
	\phantomsection
	\subsection*{{\color{purple}\underline{Problem 10}}}
	\addcontentsline{toc}{subsection}{\numberline{}Problem 10}
	\begin{enumerate}[(a)]
	\item Prove that $f(x) = x^2$ is not uniformly continuous. 
	\begin{proof}
		Recall the definition of uniformly continuous, \\
		\textbf{Definition. } The function $f$ is uniformly continuous on an 
		interval $A$ for every $\epsilon > 0$, there exists $\delta > 0$ such that
		$\forall x \in A$, if $|x - y| < \delta$ then $|f(x) - f(y)| < \epsilon$. \\
		And the negation of the definition is: \\
		The function $f$ is \emph{not} uniformly continuous on an interval $A$ means
		there exists $\epsilon > 0$, for all $\delta > 0$ such that $\exists x \in A$
		such that if $|x - y| < \delta$ then $|f(x) - f(y)| \geq \epsilon$. \\
		Now consider a fix $\epsilon > 0$, choose $\delta = \epsilon$. Let $y = x + \dfrac{\delta}{2}$, we have
		that if $|x - y| = |x - (x + \dfrac{\delta}{2})| = \bigg|\dfrac{\delta}{2}\bigg| < \delta$ 
		which is true, but 
		$$|f(x) - f(y)| = |x^2 - (x + \dfrac{\delta}{2})^2|
		|x^2 - x^2 - 2 \cdot x \dfrac{\delta}{2} - \dfrac{\delta^2}{4}| = 
		|-x\delta - \dfrac{\delta^2}{4}| = |x \epsilon + \dfrac{\epsilon^2}{4}| \geq |x \epsilon|$$
		
		Thus if $x \geq 1$ then $|f(x) - f(y)| \geq |x\epsilon| \geq \epsilon$. Therefore,
		$f(x) = x^2$ is not uniformly continuous.	
	\end{proof}
	
	\item Prove that $f(x) = \sqrt{x}$ is uniformly continuous. 
	\begin{proof}
		Fix $\epsilon > 0$, let $\delta = \epsilon^2$, then for all $x \in \mathrm{dom}(f) = [0, \infty)$.
		If $|x - y| < \epsilon^2 \Leftrightarrow x - y < \epsilon^2 \Leftrightarrow 
		x < \epsilon^2 + y$. \\
		On the other hand, we claim that $\sqrt{x + y} < \sqrt{x} + \sqrt{y}$. To prove
		we square both sides, $(\sqrt{x + y})^2 \leq (\sqrt{x} + \sqrt{y})^2 \Leftrightarrow
		x + y \leq x + y + 2\sqrt{x}{y} \Leftrightarrow 0 \leq 2\sqrt{x}{y}$ which is true. \\
		Now consider 
		$x < \epsilon^2 + y \Leftrightarrow \sqrt{x} < \sqrt{\epsilon^2 + y} \leq
		\sqrt{\epsilon^2} + \sqrt{y} \Rightarrow \sqrt{x} - \sqrt{y} < \epsilon \Rightarrow
		\sqrt{x} - \sqrt{y} > -\epsilon \Rightarrow |\sqrt{x} - \sqrt{y}| < \epsilon$.
		Since $\epsilon$ is arbitrary, $f(x) = \sqrt{x}$ is uniformly continuous. 
	\end{proof}		
	\end{enumerate}
	
	\phantomsection
	\subsection*{{\color{purple}\underline{Problem 11}}}
	\addcontentsline{toc}{subsection}{\numberline{}Problem 11}
	\begin{enumerate}[(a)]
	\item Prove that if $f$ and $g$ are uniformly continuous on $\mathbb{R}$, then so is $f + g$.
	\begin{proof}
		By definition of uniformly continuous, let $\epsilon > 0$, 
		$$\exists \delta_1 > 0 \text{ such that if } |x - y| < \delta_1 \Rightarrow |f(x) - f(y)| < \dfrac{\epsilon}{2}$$
		$$\exists \delta_2 > 0 \text{ such that if } |x - y| < \delta_2 \Rightarrow |g(x) - g(y)| < \dfrac{\epsilon}{2}$$
		What we want to show is that if $|x - y| < \delta = \min(\delta_1, \delta_2)$ then 
		$|(f + g)(x) - (f + g)(y)| < \epsilon$. Consider,
		$$|(f + g)(x) - (f + g)(y)| = |f(x) + g(x) - f(y) - g(y)| \leq 
		|f(x) - f(y)| + |g(x) - g(y)| < \dfrac{\epsilon}{2} + \dfrac{\epsilon}{2} = \epsilon$$
	\end{proof}
	
	\item Prove that if $f$ and $g$ are uniformly continuous and bounded on 
	$\mathbb{R}$, then $f \cdot g$ is uniformly continuous on $\mathbb{R}$.
	\begin{proof}
		Similarly to part a), but this time we want our $\epsilon$ slightly different because we have
		to take into account the bounded property. In fact, if we expand,
		\begin{eqnarray*}		
		|(f \cdot g)(x) - (f \cdot g)(y)| &=& |f(x)g(x) - f(y)g(y)| \\
		&=& |f(x)g(x) - g(x)f(y) + g(x)f(y) - f(y)g(y)| \\
		&=& |g(x)[f(x) - f(y)] + f(y)[g(x) - g(y)]|\\
		& \leq & |g(x)[f(x) - f(y)]| + |f(y)[(g(x) - g(y)]|\\ 
		& \leq & |g(x)|(f(x) - f(y))| + |f(y)|(g(x) - g(y))| \\ 
		& < & M \dfrac{\epsilon}{2M} + M \dfrac{\epsilon}{2M} = \epsilon
		\end{eqnarray*}
		where $M$ is the bound for both $f$ and $g$, specifically $M = \max(M_1, M_2)$.
	\end{proof}

	\item Show that the conclusion in $(a)$ above does not hold if one of the function $f$ or $g$ is not bounded.	
	\begin{proof}
		Consider the last inequality
		$$|g(x)|(f(x) - f(y))| + |f(y)|(g(x) - g(y))| < \epsilon$$
		If either $g$ or $f$ is unbounded, without loss of generality assume $g(x)$ is unbounded and it's 
		increasing, then no matter what $\epsilon$ we choose, there will be some $x$ such that $
		|g(x)| > \epsilon$ because it's unbounded! Thus, $(a)$ will not hold if one of the functions is
		not bounded.
	\end{proof}		
	\end{enumerate}
	
	\phantomsection
	\subsection*{{\color{purple}\underline{Problem 12}}}
	\addcontentsline{toc}{subsection}{\numberline{}Problem 12}
	Let $f$ be uniformly continuous. Prove that if $(x_n)$ is a Cauchy sequence in 
	$\mathrm{dom}(f)$, 
	then $(f(x_n))$ is also a Cauchy sequence. Show by counterexample that the condition 
	"uniformly" is necessary.
	\begin{proof}
	Recall, \\
	\textbf{Cauchy Sequence. } A sequence $X = (x_n)$ of real numbers is said to be
	a Cauchy Sequence if for every $\epsilon > 0$, there exists a natural number $N$
	such that for all natural numbers $m, n \geq N$, then $|x_n - x_m| < \epsilon$. \\
	
	\textbf{Uniformly continuous. } The function $f$ is said to be uniformly continuous if
	$\forall \epsilon > 0$, $\exists \delta < 0$ such that $|f(x) - f(y)| < \epsilon$
	for every $x, y \in \mathrm{dom}(f)$ satisfying $|x - y| < \delta$. 
	
	From the hypothesis there exists a Cauchy sequence $(x_n)$ in $\mathrm{dom}(f)$.
	By definition of Cauchy sequence, fix $\epsilon > 0$, there exists $N \in \mathrm{N}$
	such that for all natural numbers $m, n \geq N$, then $|x_n - x_m| < \epsilon$.
	What we want to show is that, for all $\gamma > 0$,
	\begin{equation}
		|f(x_n) - f(x_m)| < \gamma \text{,   if } |x_n - x_m| < \epsilon 
	\end{equation}
	On the other hand, since $f$ is uniformly continuous, $(1)$ holds. Thus
	$(f(x_n))$ is also a Cauchy sequence. \\
	The uniformly continuous condition is necessary, otherwise we could have,
	$$\mathrm{dom}(f) = \{\dfrac{1}{n}: n \in \mathrm{N}\}$$
	We claim that $\dfrac{1}{n}$ is a Cauchy sequence. In fact, given $\epsilon > 0$,
	we choose $N = \dfrac{2}{\epsilon}$, then if $m, n > N$, we have
	$\dfrac{1}{n} \leq \dfrac{1}{N} < \dfrac{\epsilon}{2}$ and 
	$\dfrac{1}{m} \leq \dfrac{1}{N} < \dfrac{\epsilon}{2}$, then
	$$\bigg|\dfrac{1}{n} - \dfrac{1}{m}\bigg| < \dfrac{\epsilon}{2} + \dfrac{\epsilon}{2} = \epsilon$$
	Since $\epsilon$ is arbitrarily chosen, $\mathrm{dom}(f)$ contains a Cauchy sequence.
	Now to find a function $f$ that is \emph{not} "uniformly" continuous for this domain, we can just choose
	$f$ to be:
	$$
	f(x) = 
	\begin{cases}
		1 &, \text{ if } n \text{ is odd } \\
		0 &, \text{ if } n \text{ is even }
	\end{cases}
	$$
	Apparently, $f$ is continuous but it's not uniformly continuous because no matter what
	$\delta > 0$ we choose to satisfy $|x - y| < \delta$, $|f(x) - f(y)| = 1 > \epsilon$ for $\epsilon = \dfrac{1}{2}$ for example. Therefore we can conclude that the condition "uniformly" is essential 
	for $(f(x_n))$ to be a Cauchy sequence.
	\end{proof}
	
	\phantomsection
	\subsection*{{\color{purple}\underline{Problem 13}}}
	\addcontentsline{toc}{subsection}{\numberline{}Problem 13}
	We proved in class that a continuous function attains a maximum value on any compact subset
	of its domain. Find the maximum value of $f(x) = x^3 - 9x$ in the interval $[-3,3]$. 
	\textbf{Note}: No derivatives!
	
	
	\phantomsection
	\subsection*{{\color{purple}\underline{Problem 14}}}
	\addcontentsline{toc}{subsection}{\numberline{}Problem 14}
	Let $f: \mathbb{R} \rightarrow \mathbb{R}$ be the function given by $f(x) = x$ if $x$ is rational and $f(x) = -x$ if $x$ is irrational.
	Prove that $f$ is continuous at $p = 0$.
	\begin{proof}
	$f$ is continuous at $0$. Given $\epsilon > 0$, take $\delta = \epsilon$. If $|x| < \delta$, then
	$$|f(x) - f(0)| = |f(x)| = |x| < \delta = \epsilon$$
	If $p \neq 0$ then $f$ is not continuous at $p$. Indeed, if $p$ is not rational there is a sequence $(p_n)$ of rational numbers such that
	$p_n \rightarrow p$, and if $p$ is rational, there is a sequence of irrational numbers $p_n \rightarrow p$ (for example, 
	$p_n = p + \sqrt{2}/n$. In either case, $\displaystyle\lim_{n\to \infty}f(p_n) = -p$ for $p \neq 0$.
	\end{proof}
	
	
	\phantomsection
	\subsection*{{\color{purple}\underline{Problem 15}}}
	\addcontentsline{toc}{subsection}{\numberline{}Problem 15}
	\begin{enumerate}[(a)]	
	\item For which of the following values of $\alpha$ is the function $f(x) = x^{\alpha}$ uniformly continuous on $[0, \infty): \alpha = 1/3, 1/2, 2, 3$?
	\begin{proof}
		For $y > x$, we have by Mean Value Theorem,
		$$y^{\alpha} - x^{\alpha} = (y - x)\alpha c^{\alpha - 1}, \, \, \, \, x < c < y$$
		So for $\alpha > 1$ we have
		$$y^{\alpha} - x^{\alpha} \geq \alpha x^{\alpha - 1}(y - x)$$
		Since $x^{\alpha - 1}$ is unbounded on $[0, \infty)$, we cannot make $y^{\alpha} - x^{\alpha} < \epsilon$ simply by making 
		$y - x$ less than any fixed $\delta$. So $f$ is not uniformly continuous on $[0, \infty)$ for $\alpha > 1$. For $0 < \alpha < 1$ we have
		to be a little more careful, we have
		$$y^{\alpha} - x^{\alpha} \leq \alpha y^{\alpha - 1}(y - x) \leq \alpha (y - x), \, \, \, \, \text{ for } y \geq 1$$
		which at least show that $f$ is uniformly continuous on $[1, \infty)$. Since it is also uniformly continuous on $[0, 1]$,
		it follows that it is uniformly continuous on $[0, \infty)$. 
	\end{proof}

	\item Find a function $f$ that is continuous and bounded on $(0, 1]$ but not uniformly continuous on $(0, 1]$.
	\begin{proof}
		$f(x) = \sin(1/x)$
	\end{proof}
	\end{enumerate}
	
	\phantomsection
	\subsection*{{\color{purple}\underline{Problem 16}}}
	\addcontentsline{toc}{subsection}{\numberline{}Problem 16}
	\begin{enumerate}[(a)]
	\item Prove that the intersection of arbitrary family of compact sets is compact.
	\begin{proof}
	Let $\{K_i\}_{i\in I}$ be a family of compact sets, and let 
	$$K = \displaystyle\bigcap_{i \in I} K_i$$ denote their intersection. We will show that
	$K$ is compact by showing that it is closed and bounded. \\
	In fact, each $K_i$ is bounded because it is compact and $K \subset K_i$ for all $i$ so
	$K$ must be bounded as any bound for $K_i$ will also be a bound for $K$. \\
	The set $K$ is closed because the intersection of closed sets is a closed set.
	\end{proof}
	
	\item Prove that the union of finitely compact sets is compact.
	\begin{proof}
	Suppose that $K_1$ and $K_2$ are compact, and let $K = K_1 \cup K_2$ be their union.
	Let $(x_n)$ be a sequence in $K$. Each $x_n$ is in one of the two sets $K_1$ or $K_2$ (it could be in both),
	so it follows that there is a subsequence $(x_{n_m})$ of $(x_n)$ where all terms $x_{n_m}, m = 1, 2, 3, \ldots$
	are in the same $K_i, i = 1 or 2$. Since $K_i$ is compact, this subsequence $(x_{n_m})$ has a subsequence $(x_{n_{m_l}})$
	which converges to a point $x$ in $K_i$. But $(x_{n_{m_l}})$ is also a subsequence of original sequence $(x_n)$ and its 
	limit $x$ is in $K \supset K_i$. \\
	This proves that the union of two compact sets is compact. For finite unions, the proof proceeds by induction
	on the number of sets. Suppose that you have proved that the union of $< n$ compact sets is a compact.
	If $K_1, K_2, \ldots K_n$ is a collection of $n$ compact sets, then their union can be written as $K
	 = K_1 \cup (K_2 \cup \ldots K_n)$, the union of two compact sets, hence compact.
	\end{proof}
	\end{enumerate}
	
	\phantomsection
	\subsection*{{\color{purple}\underline{Problem 17}}}
	\addcontentsline{toc}{subsection}{\numberline{}Problem 17}
	
	
	
	
	
	
	
	
	
	
\end{document}
