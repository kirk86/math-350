\documentclass[10pt,letterpaper]{article}
\renewcommand{\rmdefault}{ptm}

\usepackage[left=1in,right=1in,top=1in,bottom=1in]{geometry} 
\usepackage{amsmath}
\usepackage{amsfonts}
\usepackage{amsthm}
\usepackage{amssymb}
\usepackage{polynomial}
\usepackage{layouts}
\usepackage{enumerate}
\usepackage{syntax}
\usepackage{gensymb}
\usepackage{cancel}
\usepackage{calc}
\usepackage{enumerate}
\usepackage{xcolor}

\usepackage{minted}

\usepackage[version=0.96]{pgf}
\usepackage{tikz}
\usetikzlibrary{arrows,shapes,automata,backgrounds,petri,positioning}
\usetikzlibrary{decorations.pathmorphing}
\usetikzlibrary{decorations.shapes}
\usetikzlibrary{decorations.text}
\usetikzlibrary{decorations.fractals}
\usetikzlibrary{decorations.footprints}
\usetikzlibrary{shadows}
\usetikzlibrary{calc}
\usetikzlibrary{spy}
\usetikzlibrary{matrix}

\usepackage{tikz-qtree}

\setcounter{tocdepth}{2}
\setcounter{secnumdepth}{4}
\usepackage[bookmarksopen,bookmarksdepth=3]{hyperref}
\usepackage{titlesec}


%define new colors
\definecolor{dark-red}{rgb}{0.8,0.15,0.15}
\definecolor{dark-blue}{rgb}{0.15,0.15,0.7}
\definecolor{medium-blue}{rgb}{0,0,0.5}

%set up color for table of contents
\hypersetup{
    colorlinks, linkcolor={dark-red},
    citecolor={dark-blue}, urlcolor={medium-blue}
}

\usepackage{tocloft}

%preven linebreak between subsection header and its content
\titleformat{\subsection}[runin]{\normalfont\bfseries}{\thesubsection.}{2pt}{}
%\titleformat{\section}[runin]{\normalfont\bfseries\filcenter}{\thesection.}{5pt}{}


\titleformat{\section}[block]
{\normalfont\sffamily\LARGE}
{\thesection}{.2em}{\titlerule\\[.2ex]\bfseries}

%title
\title{\textbf{Math 350 - Advanced Calculus \\ Homework 2}}
\author{Chan Nguyen}

%set numwidth of section
\setlength{\cftsecnumwidth}{1.5cm} 
%make subsection numwidth different than as section
\setlength{\cftsubsecnumwidth}{3cm}
%make subsection indent the same as section
\setlength{\cftsubsecindent}{\cftsecindent} 

\newcommand{\sol}{\emph{\textbf{Solution. }}}

\usepackage{tikz}
\usetikzlibrary{matrix}
\usetikzlibrary{shapes,backgrounds}

\makeatletter
\newcommand{\DESCRIPTION@original@item}{}
\let\DESCRIPTION@original@item\item
\newcommand*{\DESCRIPTION@envir}{DESCRIPTION}
\newlength{\DESCRIPTION@totalleftmargin}
\newlength{\DESCRIPTION@linewidth}
\newcommand{\DESCRIPTION@makelabel}[1]{\llap{#1}}%
\newcommand{\DESCRIPTION@item}[1][]{%
  \setlength{\@totalleftmargin}%
       {\DESCRIPTION@totalleftmargin+\widthof{\textbf{#1 }}-\leftmargin}%
  \setlength{\linewidth}
       {\DESCRIPTION@linewidth-\widthof{\textbf{#1 }}+\leftmargin}%
  \par\parshape \@ne \@totalleftmargin \linewidth
  \DESCRIPTION@original@item[\textbf{#1}]%
}
\newenvironment{DESCRIPTION}
  {\list{}{\setlength{\labelwidth}{0cm}%
           \let\makelabel\DESCRIPTION@makelabel}%
   \setlength{\DESCRIPTION@totalleftmargin}{\@totalleftmargin}%
   \setlength{\DESCRIPTION@linewidth}{\linewidth}%
   \renewcommand{\item}{\ifx\@currenvir\DESCRIPTION@envir
                           \expandafter\DESCRIPTION@item
                        \else
                           \expandafter\DESCRIPTION@original@item
                        \fi}}
  {\endlist}
\makeatother

\begin{document}

\tableofcontents 
\maketitle

\setlength{\parindent}{0pt}
\setlength{\parskip}{1ex}
	\phantomsection
	\subsection*{{\color{purple}\underline{Problem 1}}}
	\addcontentsline{toc}{subsection}{\numberline{}Problem 1}
	Let $I$ be a set and for each $i \in I$, let $X_i$ be another set. We may speak of $I$ as being an
	indexing set, whose elements, $i$, are indexes used to specify the sets $X_i$ which we direct our attention. 
	The set of all sets $X_i$ as $i$ ranges over $I$ is denoted by $\{X_i \mid i \in I\}$, or $\{X_i\}_{i \in I}$
	If all $X_i$ are subsets of a set $S$, let $\displaystyle{\bigcap_{i \in I}}X_i$ and $\displaystyle{\bigcup_{i \in I}X_i}$ be the subsets of $S$
	given by:
	$$\displaystyle{\bigcup_{i \in I}X_i} = \{x \in S \mid x \in X_i \text{ for some } i \in I\}$$
	$$\displaystyle{\bigcap_{i \in I}X_i} = \{x \in S \mid x \in X_i \text{ for all } i \in I\}$$ 	
	Prove the following:
	\begin{enumerate}[(i)]
		\item $S \setminus \displaystyle{\bigcup_{i \in I}}X_i = \displaystyle{\bigcap_{i \in I}}(S \setminus X_i)$ 
\begin{proof}
	Since all $X_i$ are subsets of $S$, we can write it as,
		$$\bigg(\displaystyle{\bigcup_{i \in I}}X_i\bigg)^c = \displaystyle{\bigcap_{i \in I}}X_i^c$$
		Suppose that $x \in \bigg(\displaystyle{\bigcup_{i \in I}}X_i\bigg)^c \implies
		x \not\in \displaystyle{\bigcup_{i \in I}}X_i \implies 
		x \not\in X_1 \text{ and } x \not\in X_2 \ldots \text{ and } x \not\in X_i, i \in I \implies 
		x \in X_1^c \text{ and } x \in X_2^c \ldots \text{ and } x \in X_i^c, i \in I$. In other words,
		$x \in \displaystyle{\bigcap_{i \in I}}X_i^c (1)$ \\
		To go the other way, suppose that $x \in \displaystyle{\bigcap_{i \in I}}X_i^c 
		\implies (x \in X_1^c) \cap (x \in X_2^c) \cap \ldots \cap (x \in X_i^c), i \in I
		\implies (x \not\in X_1) \cup (x \not\in X_2) \cup \ldots \cup (x \not\in X_i), i \in I
		\implies x \not\in \displaystyle{\bigcup_{i \in I}X_i} \implies 
		x \in \bigg(\displaystyle{\bigcup_{i \in I}X_i}\bigg)^c (2)$ \\
		From (1) and (2) we can conclude that 
		$$\bigg(\displaystyle{\bigcup_{i \in I}}X_i\bigg)^c = \displaystyle{\bigcap_{i \in I}}X_i^c$$
\end{proof}
		\item $S \setminus \displaystyle{\bigcap_{i \in I}}X_i = \displaystyle{\bigcup_{i \in I}}(S \setminus X_i)$ 
\begin{proof}
	Similarly, we can rewrite it as,
		$$\bigg(\displaystyle{\bigcap_{i \in I}}X_i\bigg)^c = \displaystyle{\bigcup_{i \in I}}X_i^c$$
		Suppose that $x \in \bigg(\displaystyle{\bigcap_{i \in I}}X_i\bigg)^c \implies 
		x \not\in \displaystyle{\bigcap_{i \in I}}X_i \implies 
		x \in \displaystyle{\bigcup_{i \in I}}X_i^c$ (1) \\
		To go the other way, suppose that $x \in \displaystyle{\bigcup_{i \in I}}X_i^c \implies
		(x \in X_1^c) \cup (x \in X_2^c) \cup \ldots \cup (x \in X_i^c) \implies 
		(x \not\in X_1) \cup (x \not\in X_2) \cup \ldots \cup (x \not\in X_i) \implies
		x \not\in \displaystyle{\bigcap_{i \in I}}X_i \implies
		x \in \bigg(\displaystyle{\bigcap_{i \in I}}X_i\bigg)^c 
		$ (2) \\
		From (1) and (2) we can conclude that
		$$\bigg(\displaystyle{\bigcap_{i \in I}}X_i\bigg)^c = \displaystyle{\bigcup_{i \in I}}X_i^c$$
\end{proof}
		\item $\displaystyle{\bigcup_{i \in I}}X_i \cap \displaystyle{\bigcup_{j \in J}}Y_j 
		= \displaystyle{\bigcup_{(i, j) \in I \times J}}(X_i \cap Y_j)$ 
\begin{proof}
	Suppose that $x \in \bigg(\displaystyle{\bigcup_{i \in I}}X_i \cap \displaystyle{\bigcup_{j \in J}}Y_j\bigg)
		\implies x \in
		\bigg((X_1 \cup X_2 \cup \ldots \cup X_i) \cap (Y_1 \cup Y_2 \cup \ldots \cup Y_i)\bigg)
		\implies x \in
		\bigg([X_1 \cap (Y_1 \cup Y_2 \cup \ldots \cup Y_i)] \cup [X_2 \cap (Y_1 \cup Y_2 \cup \ldots \cup Y_i)] 
		\ldots [X_i \cap (Y_1 \cup Y_2 \cup \ldots \cup Y_i)] \bigg) \implies
		x \in \displaystyle{\bigcup_{(i, j) \in I \times J}}(X_i \cap Y_j) 	
		$ (1) \\
		To go the other way around, suppose that $x \in \displaystyle{\bigcup_{(i, j) \in I \times J}}(X_i \cap Y_j) 
		\implies x \in
		\bigg([X_1 \cap (Y_1 \cup Y_2 \cup \ldots \cup Y_i)] \cup [X_2 \cap (Y_1 \cup Y_2 \cup \ldots \cup Y_i)] 
		\ldots [X_i \cap (Y_1 \cup Y_2 \cup \ldots \cup Y_i)] \bigg) \implies 
		x \in \bigg(\displaystyle{\bigcup_{i \in I}}X_i \cap \displaystyle{\bigcup_{j \in J}}Y_j\bigg) 		
		$ (2) \\
		From (1) and (2) we can conclude that 
		$$\displaystyle{\bigcup_{i \in I}}X_i \cap \displaystyle{\bigcup_{j \in J}}Y_j 
		= \displaystyle{\bigcup_{(i, j) \in I \times J}}(X_i \cap Y_j)$$
\end{proof}		
		
		\item $\displaystyle{\bigcap_{i \in I}}X_i \cup \displaystyle{\bigcap_{j \in J}}Y_j 
		= \displaystyle{\bigcap_{(i, j) \in I \times J}}(X_i \cup Y_j)$ 
\begin{proof}
	Similarly, suppose that $x \in \bigg(\displaystyle{\bigcap_{i \in I}}X_i \cup \displaystyle{\bigcap_{j \in J}}Y_j\bigg)
		\implies x \in \bigg((X_1 \cap X_2 \cap \ldots \cap X_I) \cup (Y_1 \cap Y_2 \cap \ldots \cap Y_i)\bigg)
		\implies x \in
		\bigg(
		[X_1 \cup (Y_1 \cap Y_2 \cap \ldots \cap Y_i)] \cap  		
		[X_2 \cup (Y_1 \cap Y_2 \cap \ldots \cap Y_i)] \cap
		\ldots
		[X_i \cup (Y_1 \cap Y_2 \cap \ldots \cap Y_i)]
		\bigg) \implies 
		x \in \bigg(\displaystyle{\bigcap_{(i, j) \in I \times J}}(X_i \cup Y_j)\bigg)
		$ (1) \\
		To go the other way, suppose that $x \in \bigg(\displaystyle{\bigcap_{(i, j) \in I \times J}}(X_i \cup Y_j)\bigg)
		\implies  
		x \in
		\bigg(
		[X_1 \cup (Y_1 \cap Y_2 \cap \ldots \cap Y_i)] \cap  		
		[X_2 \cup (Y_1 \cap Y_2 \cap \ldots \cap Y_i)] \cap
		\ldots
		[X_i \cup (Y_1 \cap Y_2 \cap \ldots \cap Y_i)]
		\bigg)
		\implies x \in \bigg((X_1 \cap X_2 \cap \ldots \cap X_I) \cup (Y_1 \cap Y_2 \cap \ldots \cap Y_i)\bigg)
		\implies x \in \bigg(\displaystyle{\bigcap_{i \in I}}X_i \cup \displaystyle{\bigcap_{j \in J}}Y_j\bigg)$ (2)
		From (1) and (2) we can conclude that
		$$\displaystyle{\bigcap_{i \in I}}X_i \cup \displaystyle{\bigcap_{j \in J}}Y_j 
		= \displaystyle{\bigcap_{(i, j) \in I \times J}}(X_i \cup Y_j)$$
\end{proof}
	\end{enumerate}
	
	\phantomsection
	\subsection*{{\color{purple}\underline{Problem 2}}}
	\addcontentsline{toc}{subsection}{\numberline{}Problem 2}
	Prove that if a set $A$ is enumerable (at most countable), then either $A = \emptyset$ or there is a
	surjective mapping $f: \mathbf{N} \rightarrow A$. 
\begin{proof}
	First we recall 4 properties of set:
	\begin{enumerate}
		\item Two sets are equipotent if there is a bijection between them. To be equipotent
		is an equivalence relation among sets. 
		\item A set is finite if it's equipotent to one of the finite sets $\mathbf{0}, \mathbf{1}, \ldots$.
		\item A set is countable if it is equipotent to the set $\mathbf{N}$ for all natural numbers. 
		\item A set is enumerable if it is either finite or countable.
	\end{enumerate}
	Essentially, we know that $A$ is either finite or countable, so we have two cases:
	\begin{itemize}
		\item Assume that $A$ is finite then $A$ is equipotent to one of the sets $\mathbf{0}, \mathbf{1}, \ldots$.
		Next let's assume that $A$ is equipotent to $\mathbf{0} = \emptyset$, thus $A$ must equal to $\emptyset$.
		\item Assume that $A$ is countable, then $A$ is equipotent to $\mathbf{N}$ which implies there is a bijection between 
		$A$ and $\mathbf{N}$. Thus there is a surjective mapping $f: \mathbf{N} \rightarrow A$
	\end{itemize}
	$\therefore A = \emptyset$ or there is a surjective mapping $f: \mathbf{N} \rightarrow A$.
\end{proof}
	
	\phantomsection
	\subsection*{{\color{purple}\underline{Problem 3}}}
	\addcontentsline{toc}{subsection}{\numberline{}Problem 3}
	\text{ }
	\begin{enumerate}[(i)]
\item Prove that if $\{X_i \mid i \in I\}$ is a set of sets where the indexing set $I$ is countable
		and each $X_i$ is countable, then $\displaystyle{\bigcup_{i \in I}}X_i$ is countable.

		To be precise, first we recall the definition of countable set. What does it mean to count 
		the number of elements in the set? What we really do we count is to assign each element of the 
		set a unique natural number, either starting from 0 or 1. and proceeding upward. For example,
		count the set of lower-case letters in alphabet:
		$$S = \{a, b, c, \ldots, x, y, z\}$$
		\begin{center}
		\begin{tabular}{ccccccc}
			$a$ & $b$ & $c$ & $\ldots$ & $x$ & $y$ & $z$ \\
			$\updownarrow$ & $\updownarrow$ & $\updownarrow$ & $\ldots$ & $\updownarrow$ & $\updownarrow$ & $\updownarrow$ \\
			$1$ & $2$ & $3$ & $\ldots$ & $24$ & $25$ & $26$ 
		\end{tabular}
		\end{center}
		In fact, when we count we are using a function from our set $S$ to a subset of the natural number $\mathbf{N}$
		that is both one-to-one and onto. \\
		\begin{itemize}
			\item A function is \emph{one-to-one} if two distinct inputs always have two distinct outputs. 
			\item A function is \emph{onto} if every element in the range is the output of one or more elements
			in the domain.
			\item \emph{bijection} = \emph{one-to-one} + \emph{onto}. 
		\end{itemize}
	
	\begin{proof}
		First we have that the indexing set $I$ is countable, so there is a mapping from $\mathbf{N}$ to $I$,
		symbolically,
		$$f: \mathbf{N} \rightarrow I$$
		so we can label each element $x \in I$ with the natural number $1, 2, 3, \ldots$ 
		Let denote all countable sets $X_i$ as:
		$$X_1, X_2, X_3, X_4 \ldots$$
		What we want to show is the union of these sets is also countable.
		$$X = \displaystyle\bigcup_{i=1}^{\infty} X_i = X_1 \cup X_2 \cup X_3 \cup \ldots$$
		Suppose $X_i$ has elements:
		$$a_{i1}, a_{i2}, a_{i3} \ldots$$
		Next we put these elements of the union of these countable set into a 2-dimension infinite array: \\
		\begin{center}
		\begin{tabular}{ccccc}
			$a_{11}$ & $a_{12}$ & $a_{13}$ & $a_{14}$ & $\ldots$ \\
			$a_{21}$ & $a_{22}$ & $a_{23}$ & $a_{24}$ & $\ldots$ \\
			$a_{31}$ & $a_{32}$ & $a_{33}$ & $a_{34}$ & $\ldots$ \\
			$a_{41}$ & $a_{42}$ & $a_{43}$ & $a_{44}$ & $\ldots$ \\
			$\ldots$ & $\ldots$ & $\ldots$ & $\ldots$ & $\ldots$ 
		\end{tabular}
		\end{center}
		Now to show that this is a countable set, we need to create bijection mapping between these elements with the
		natural number sets $\mathbf{N}$ using the method as described in lecture:
	
	\begin{center}	
	\begin{tikzpicture}[shorten >=1pt,node distance=2cm,minimum size=1cm, on grid,auto] 
   	\node[] (a11)  			      {$a_{11}$};
   	\node[] (a12) [right =of a11] {$a_{12}$};
   	\node[] (a13) [right =of a12] {$a_{13}$};
   	\node[] (a14) [right =of a13] {$a_{14}$};
   	
   	\node[] (a21) [below =of a11] {$a_{21}$};
   	\node[] (a22) [right =of a21] {$a_{22}$};
   	\node[] (a23) [right =of a22] {$a_{23}$};
   	\node[] (a24) [right =of a23] {$a_{24}$};
   	
   	\node[] (a31) [below =of a21] {$a_{31}$};
   	\node[] (a32) [right =of a31] {$a_{32}$};
   	\node[] (a33) [right =of a32] {$a_{33}$};
   	\node[] (a34) [right =of a33] {$a_{34}$};                
   
   	\node[] (a41) [below =of a31] {$a_{41}$};
   	\node  		 (a42) [right =of a41] {$\ldots$};
   	\path[->] 
   	(a11) edge (a12)
   	(a12) edge (a21)
   	(a21) edge (a31)
   	(a31) edge (a22)
   	(a22) edge (a13)
   	(a13) edge (a14)
   	(a14) edge (a23)
   	(a23) edge (a32)
   	(a32) edge (a41)
   	(a41) edge (a42)
   	;
	\end{tikzpicture}
	\end{center}		
	The bijection mapping above indeed map each element in the union to the set of natural number $\mathbf{N}$ \\
	\begin{center}
	\begin{tabular}{cccccccc}
		$a_{11}$ & $a_{12}$ & $a_{21}$ & $a_{31}$ & $a_{22}$ & $a_{13}$ & $a_{14}$ & $\ldots$ \\
		$\updownarrow$ & $\updownarrow$ & $\updownarrow$ & $\updownarrow$ & $\updownarrow$ & $\updownarrow$ & $\updownarrow$ & \\
		$1$ & $2$ & $3$ & $3$ & $4$ & $5$ & $6$ & $\ldots$
	\end{tabular}
	\end{center}
	Therefore $X$ is countable.
\end{proof}	
	
\item Prove that $P(\mathbf{N})$, the set of all finite subsets of $\mathbf{N}$ is countable.
\begin{proof}
			
		Let $S = P(\mathbf{N})$ be the set of all finite subset of $\mathbf{N}$. In addition, 
		we have that if a set is finite, then it is countable. Now let $S_k$ be the set of subsets of $\mathbf{N}$
		that consists of $k$ elements, we can reorder these $k$ elements in increasing order as follows:
		$$x_0 < x_1 < x_2 < \ldots x_k$$
		Let define the sum and sum square of all elements in $S_k$ as:
		$$s = \displaystyle\sum_{i=0}^{k}x_i \text{ and } 
			sq = \displaystyle\sum_{i=0}^{k}x_i^2$$
		It's easy to see that a triplet of $(s, sq, k)$ is unique. 
		Since $S_k$ is finite, we can list all the finite subsets of $\mathbf{N}$ as:
	\end{proof}
\end{enumerate}
	
	\phantomsection
	\subsection*{{\color{purple}\underline{Problem 4}}}
	\addcontentsline{toc}{subsection}{\numberline{}Problem 4}
	Prove the following:
	\begin{enumerate}[(i)]
	\item $a \cdot c^{-1} = (a \cdot b) \cdot (b \cdot c)^{-1}$ if $b, c \neq 0$ 
	\item $a \cdot b^{-1} + c \cdot d^{-1} = (a \cdot d + b \cdot c) \cdot (b \cdot d)^{-1}$ if $b, d \neq 0$ 
	\item $(a \cdot b)^{-1} = a^{-1} \cdot b^{-1}$ if $a, b \neq 0$ 
	\item $ (a \cdot b^{-1}) \cdot (c \cdot d^{-1}) = (ac) \cdot (db)^{-1}$ if $b, d \neq 0$
	\item $(a \cdot b^{-1}) \cdot (c \cdot d^{-1})^{-1} = (a \cdot d) \cdot (b \cdot c)^{-1}$ if $b, c, d \neq 0$
	\item If $b, d \neq 0$, then $a \cdot b^{-1} = c \cdot d^{-1}$ if and only if $a \cdot d = b \cdot c$
	\end{enumerate}
\begin{proof}
	\text{ }
	\begin{enumerate}[(i)]
	\item  Multiply both sides by $(b \cdot c)$ to obtain, 
	\begin{eqnarray*}
		& & a \cdot c^{-1} (b \cdot c) = (a \cdot b) \cdot (b \cdot c)^{-1} \cdot (b \cdot c) \\
		&\Leftrightarrow & a \cdot c^{-1} (c \cdot b) = (a \cdot b) \cdot 1 \text{ (multiplicative inverse, associative)} \\
		&\Leftrightarrow & a \cdot (c^{-1} \cdot c) \cdot b = a \cdot b \text{ (associative, neutral element)} \\
		&\Leftrightarrow & (a \cdot 1) \cdot b = a \cdot b \text{ (multiplicative inverse)}\\
		&\Leftrightarrow & a \cdot b = a \cdot b \text{ (neutral element)}\\
	\end{eqnarray*}
	\item From (i) we have
	\begin{eqnarray*}
		& &  a \cdot b^{-1} + c \cdot d^{-1} = (a \cdot d + b \cdot c) \cdot (b \cdot d)^{-1} \\
		&\Leftrightarrow & (a \cdot d) \cdot (d \cdot b)^{-1} + (c \cdot b) \cdot (b \cdot d)^{-1} =
		(a \cdot d + b \cdot c) \cdot (b \cdot d)^{-1}
		\text{ (from (i))} \\
		&\Leftrightarrow & (a \cdot d + b \cdot c) \cdot (b \cdot b)^{-1} =
		(a \cdot d + b \cdot c) \cdot (b \cdot d)^{-1}
		\text{ (associative, distributive)} \\
	\end{eqnarray*}
	\item Multiply both side by $(a \cdot b)$ to obtain
	\begin{eqnarray*}
		& & (a \cdot b)^{-1} = a^{-1} \cdot b^{-1} \\
		& \Leftrightarrow & (a \cdot b)^{-1} \cdot (a \cdot b) = a^{-1} \cdot b^{-1} \cdot (a \cdot b) \\
		& \Leftrightarrow & 1 = (a^{-1} \cdot a) \cdot (b^{-1} \cdot b) 
		\text{ (multiplicative inverse, associative)}\\
		& \Leftrightarrow & 1 = 1 \cdot 1 
		\text{ (multiplicative inverse)}\\
		& \Leftrightarrow & 1 = 1 
		\text{ (neutral element)}\\
	\end{eqnarray*}
	
	\item We have
	\begin{eqnarray*}
	& & (a \cdot b^{-1}) \cdot (c \cdot d^{-1}) = (ac) \cdot (db)^{-1} \\
	& \Leftrightarrow & (ac) \cdot (b^{-1} \cdot d^{-1}) = (ac) \cdot (db)^{-1} 
	\text{ (associative) }\\
	& \Leftrightarrow & (ac) \cdot (bd)^{-1} = (ac) \cdot (db)^{-1} 
	\text{ (from (iii)) }\\
	\end{eqnarray*}
	
	\item First we prove that $(d^{-1})^{-1} = d$. Multiply both sides by $d^{-1}$ to obtain:
	\begin{eqnarray*}	
	& & (d^{-1})^{-1} \cdot d^{-1} = d \cdot d^{-1} \\
	& \Leftrightarrow & 1 = 1 \text{ (multiplicative inverse)} \\
	\end{eqnarray*}
	Now we have,
	\begin{eqnarray*}
	& & (a \cdot b^{-1}) \cdot (c \cdot d^{-1})^{-1} = (a \cdot d) \cdot (b \cdot c)^{-1} \\
	& \Leftrightarrow & (a \cdot b^{-1}) \cdot (c^{-1} \cdot (d^{-1})^{-1}) = (a \cdot d) \cdot (b \cdot c)^{-1}
	\text{ (from (iii)) }\\
	& \Leftrightarrow & (a \cdot b^{-1}) \cdot (c^{-1} \cdot d) = (a \cdot d) \cdot (b \cdot c)^{-1}
	\text{ (from proof above)) }\\
	& \Leftrightarrow & (a \cdot d) \cdot (b^{-1} \cdot c^{-1}) = (a \cdot d) \cdot (b \cdot c)^{-1}
	\text{ (associative)) }\\
	& \Leftrightarrow & (a \cdot d) \cdot (b \cdot c)^{-1} = (a \cdot d) \cdot (b \cdot c)^{-1}
	\text{ (from (iii)) }\\
	\end{eqnarray*}
	
	\item We have, \\
	$\Rightarrow$: If $a \cdot b^{-1} = c \cdot d^{-1}$ where $b, d \neq 0$, then there exists 
	multiplicative inverse of $b$ and $d$. Next we multiply both sides by $bd$ to obtain:
	$a \cdot b^{-1} \cdot (bd) = c \cdot d^{-1} (bd) \Leftrightarrow
	a \cdot d = c \cdot b$ \\
	$\Leftarrow$: If $a \cdot d = b \cdot c$ where $b, d \neq 0$, then there exists a multiplicative
	of $b$ and $d$. Again, we multiply both sides of this equality by $(bd)^{-1}$ to obtain, 
	$a \cdot d \cdot d^{-1} \cdot b^{-1} = b \cdot c \cdot b^{-1} \cdot d^{-1}$. Using associative and 
	multiplicative inverse property we reduce it to: $a \cdot b^{-1} = c \cdot d^{-1}$	
	\end{enumerate}
\end{proof}
	
	\phantomsection
	\subsection*{{\color{purple}\underline{Problem 5}}}
	\addcontentsline{toc}{subsection}{\numberline{}Problem 5}
	Find all the numbers $x$ such that:
	\begin{enumerate}[(i)]
		\item $x^2 - 3x - 16 < 2$ 
		\item $\dfrac{1}{x + 1} + \dfrac{1}{x + 2} < 0$ 
		\item $\dfrac{x - 1}{x + 1} > 0$
	\end{enumerate}
	
\begin{proof}
	\text{ }
	\begin{enumerate}[(i)]
		\item 
		\begin{eqnarray*}
		& & x^2 - 3x - 16 < 2 \\
		& \Leftrightarrow & x^2 - 3x - 18 < 0 \\
		& \Leftrightarrow & (x + 3)(x - 6) < 0 \\
		\end{eqnarray*}
		We have two cases:
		$$
		\begin{cases}
			x + 3 < 0 \text{ and } x - 6 > 0 \rightarrow \text{ no solution }\\
			x + 3 > 0 \text{ and } x - 6 < 0 \rightarrow x \in (-3, 6)\\
		\end{cases}		
		$$
		$$\therefore x \in (-3, 6)$$
			
		\item
		\begin{eqnarray*}
		& & \dfrac{1}{x + 1} + \dfrac{1}{x + 2} < 0 \\
		& \Leftrightarrow & \dfrac{(x + 2) + (x + 1)}{(x + 1)(x + 2)} < 0 \\
		& \Leftrightarrow & \dfrac{2x + 3}{(x + 1)(x + 2)} < 0 \\
		\end{eqnarray*} 
		We have \\
		\begin{tabular}{c|ccccccc}
			x & & -2 & & -3/2 & & -1 & \\
			\hline
			y & $-$ & \text{und.} & $+$ & 0 & $-$ & \text{und.} & $+$ \\
		\end{tabular}
		$$\therefore x \in (-\infty, -2) \cup (-3/2, -1)$$
		\item We have \\
		\begin{tabular}{c|ccccc}
			x & & -1 & & 1 & \\
			\hline
			y & $+$ & \text{und.} & $-$ & 0 & $+$ \\
		\end{tabular}
		$$\therefore x \in (-\infty, -1) \cup (0, \infty)$$
	\end{enumerate}
\end{proof}

	\phantomsection
	\subsection*{{\color{purple}\underline{Problem 6}}}
	\addcontentsline{toc}{subsection}{\numberline{}Problem 6}
	Prove the following and write $ab$ for $a \cdot b$:
	\begin{enumerate}[(i)]
		\item If $a < b$ and $c \leq d$ then $a + c < b + d$ 
		\item If $a < b$ then $-b < -a$ 
		\item If $a < b$ and $c < d$ then $a - d < b - c$ 
		\item If $a < b$ and $0 < c$, then $ac < bc$ 
		\item If $a > 1$ then $a^2 > a$
		\item If $0 < a < 1$ then $a^2 < a$
		\item If $0 \leq a < b$ and $0 \leq c < d$ then $ac < bd$
		\item If $0 \leq a < b$ then $a^2 < b^2$ 
		\item If $a \geq 0, b \geq 0$, and $a^2 > b^2$ then $a > b$
	\end{enumerate}

\begin{proof}
	\text{ }
	\begin{enumerate}[(i)]
		\item Consider $a < b$
		\begin{eqnarray*}
		& & a < b \\
		& \Leftrightarrow & a + c < b + c \\
		& \Leftrightarrow & a + c < b + d \text{ (since } c \leq d ) \\
		\end{eqnarray*} 
		
		\item Consider $a < b$
		\begin{eqnarray*}
		& & a < b \\
		& \Leftrightarrow & a \cdot -1 < b \cdot -1\\
		& \Leftrightarrow & -b < -a \\
		\end{eqnarray*}
		
		\item Consider $c < d$
		\begin{eqnarray*}
		& & c < d \\
		& \Leftrightarrow & -d < -c\\
		& \Leftrightarrow & -d + a < -c + a \\
		& \Leftrightarrow & -d + a < -c + a < -c + b \\
		& \Leftrightarrow & a - d < b - c \\
		\end{eqnarray*}
		
		\item Consider $a < b \rightarrow b - a \in \mathbf{F_{+}}$.
		Since $\mathbf{F_{+}} \cdot \mathbf{F_{+}} \subset \mathbf{F_{+}}$, we have
		$(b - a) \cdot c \in \mathbf{F_{+}} \Leftrightarrow bc - ac \in \mathbf{F_{+}}
		\Leftrightarrow bc > ac$ by definition of "$<$".
		
		\item Consider If $a > 1$ 
		\begin{eqnarray*}
		& & a \cdot a > 1 \cdot a \\
		& \Leftrightarrow & a^2 > a \\
		\end{eqnarray*}
		
		\item Consider $0 < a < 1$ then $a^2 < a$
		\begin{eqnarray*}
		& & a < 1 \\
		& \Leftrightarrow & a \cdot a < 1 \cdot a \\
		& \Leftrightarrow & a^2 < a \\
		\end{eqnarray*}
		
		\item Consider $a < b$ 
		\begin{eqnarray*}
		& & a < b \\
		& \Leftrightarrow & a \cdot c < b \cdot c \\
		& \Leftrightarrow & a \cdot c < b \cdot c < b \cdot d \\
		& \Leftrightarrow & a \cdot c < b \cdot d \\
		\end{eqnarray*}
		
		\item Consider $a < b$
		\begin{eqnarray*}
		& & a < b \\
		& \Leftrightarrow & a \cdot a < b \cdot a \\
		& \Leftrightarrow & a \cdot a < b \cdot a < b \cdot b \\
		& \Leftrightarrow & a^2 < b^2 \\
		\end{eqnarray*}
		
		\item Consider $a^2 > b^2$ 		
		\begin{eqnarray*}
		& & a^2 > b^2 \\
		& \Leftrightarrow & a^2 - b^2 > 0 \\
		& \Leftrightarrow & (a - b)(a + b) > 0 \\
		& \Leftrightarrow & (a - b) > 0 \\
		& \Leftrightarrow & a > b \\
		\end{eqnarray*}
	\end{enumerate}
\end{proof}

	\phantomsection
	\subsection*{{\color{purple}\underline{Problem 7}}}
	\addcontentsline{toc}{subsection}{\numberline{}Problem 7}
	A \emph{Peano} system is a triple $(P, z, \sigma)$ where $P$ is a set ($\neq \emptyset$), $z \in P$, and 
	$\sigma: P \rightarrow P$ is a function such that:
	\begin{enumerate}[(a)]
		\item $\sigma(p) \neq z$ for every $p \in P$ 
		\item $\sigma$ is one to one
		\item {\color{red}If $B \subset P$ satisfies $z \in B$ and $\sigma(b) \in B$ whenever $b \in B$ then $B = P$.} \\
		For a \emph{Peano} system $(P, z, \sigma)$ prove the following:
		\begin{enumerate}[(i)]
			\item $P = \{z\} \cup \sigma(P)$ 
			\item For any $p \in P$, either $p = z$ or there is exactly one number $n = 1, 2, 3 \ldots$ such that
			$p = \overbrace{\sigma(\sigma(\sigma(\ldots(\sigma(z)))))}_{\sigma \text{ applied } n \text{ times }}$
			\item If $(Q, w, \tau)$ is any other \emph{Peano} system, then there is a bijection $f: P \rightarrow Q$
			such that $f(z) = w$ and $\tau(f(p)) = f(\sigma(p))$ for any $p \in P$
		\end{enumerate}
	\end{enumerate}
\begin{proof}
	\text{ }
	\begin{enumerate}[(i)]
		\item $P = \{z\} \cup \sigma(P)$
		Since $z \in P$ and $\sigma(p) \neq z$ for every $p \in P$, 
		we have that $\sigma(P) = P \setminus \sigma(p)$ where $\sigma(p) = z$. Thus
		$(P \setminus \{z\}) \cup \{z\} = P$ 
		\item Since $z \in P$, saying either $p \in P$ is equal to $z$ is trivial. Now suppose that
		$z \neq p$, we need to show that  there is exactly one number $n = 1, 2, 3 \ldots$ such that
			$p = \overbrace{\sigma(\sigma(\sigma(\ldots(\sigma(z)))))}_{\sigma \text{ applied } n \text{ times }}$ \\
		Let's define:
			$$\sigma(z) = x, \text{ for some } x \text{ such that } x \not\in P$$
		Next, we have
		\begin{eqnarray*}
		\sigma(z) & = & x \\
		\sigma(\sigma(z)) & = & x \\
		\end{eqnarray*}
	\end{enumerate}
\end{proof}	
	
	
	
	
	
\end{document}
